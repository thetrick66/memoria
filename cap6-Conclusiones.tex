% ---------------------------------------------------------------------------------------
\chapter{Conclusiones}\label{chap6}
Despu�s de observar el correcto funcionamiento de la automatizaci�n coordinada del sistema de dos ascensores prototipos y de la supervisi�n HMI a trav�s de Ethernet utilizando la red disponible del departamento de electr�nica, se puede dar por concluidos los objetivos propuestos de mejorar los ascensores prototipos para un mejor funcionamiento ya que son usados constantemente en el laboratorio de control industrial.\\

Adem�s se aprendi� a usar la comunicaci�n entre los dos PLC's a trav�s de protocolo PPI para poder intercambiar variables y estados para as� generar un control coordinado entre ambos ascensores.\\

Por �ltimo, la creaci�n de un HMI para supervisi�n remota v�a Ethernet donde se tuvo que empezar configurando y habilitando el m�dulo de comunicaci�n Ethernet CP243-1 IT del PLC Siemens S7-200, adem�s de generar un servidor OPC local usando el software PC Access y finalizando con el uso del software de adquisici�n de datos SCADA iFix quien permite la creaci�n de un interfaz gr�fico.
\section{Aportes de la memoria, diferencias y/o mejoras}

El principal aporte de est� memoria es el mejoramiento de los ascensores prototipo para futuros usos en el laboratorio de control industrial, ya que estos equipos son constantemente usados como experiencia del ramo con el mismo nombre del laboratorio, ya que es un sistema de eventos discretos que sirve para introducir a los alumnos al control por PLC de diferentes sistemas con caracter�sticas similares. Adem�s de entregar seguridad al sistema para no causar problemas durante el primer acercamiento de automatizaci�n, por tanto su gran diferencia a otras memorias es el mejoramiento previo de los sistemas antes de comenzar con la memoria para dejar a siguientes generaciones sistemas actualizados y de mayor grado did�ctico para aumentar el aprendizaje y familiarizaci�n con el mundo industrial.

\section{Trabajos futuros}

Siempre es necesario ir actualizando los sistemas que son usados de manera did�ctica para que sea mejor el grado de aprendizaje de los alumnos que usen este sistema, el sistema tiene muchas posibilidades para hacer futuros trabajos como puede ser agregarle un cuarto piso poniendo un sensor de piso extra y otro par de botones como los del segundo piso y as� poder darle un grado de complejidad que permita ver con claridad el correcto funcionamiento de la automatizaci�n del sistema.\\

Se puede finalizar el objetivo inconcluso de la supervisi�n remota a trav�s de una p�gina web usando Java Applets y la capacidad del m�dulo de comunicaci�n CP243-1 IT en su totalidad.\\

Por �ltimo, se pueden agregar funciones al ascensor para utilizar los m�dulos anal�gicos con los que cuenta el PLC y as� poder usar variables digitales y anal�gicas para obtener un grado de aprendizaje mucho mayor con este �nico sistema. 

