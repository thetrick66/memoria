%------------------------------------%
%			Inicio Preámbulo
%------------------------------------%
\documentclass[12pt,spanish]{thesis}
% Mayor compatibilidad y preferencia personal
%\usepackage[latin1]{inputenc}
\usepackage[utf8]{inputenc}
\usepackage[spanish]{babel}
\usepackage{amsmath}

% Interlineado
\usepackage{setspace}
\setstretch{1.5}

% Paquetes
\usepackage{textcomp}
\usepackage{times}
\usepackage{amssymb}
\usepackage{float}
\usepackage{color}
\usepackage{graphicx}
\usepackage{eso-pic}
\usepackage{multicol}
\usepackage{enumerate}
\usepackage{url}
\usepackage{soul}
\usepackage{fancyhdr}
\usepackage{lscape}
\usepackage{pdfpages}
\usepackage{hyperref}
\usepackage{listings}


% Márgenes
\usepackage[top=3cm,bottom=3cm,left=4.2cm,right=3cm]{geometry}
\pagestyle{empty} 
\frenchspacing
\fancyhead[L]{}
\fancyhead[C]{}
\fancyhead[R]{}
\fancyfoot[R]{\thepage}
\fancyfoot[C]{}

%------------------------------------%
%			Fin Preámbulo
%------------------------------------%


\begin{document}
\thispagestyle{empty}

\begin{center}
%\linespread{1.15}
\renewcommand{\baselinestretch}{1.15}
\textbf{\large{UNIVERSIDAD TÉCNICA FEDERICO SANTA MARÍA\\}
\normalsize{DEPARTAMENTO DE ELECTRÓNICA\\VALPARAÍSO - CHILE\\}}

\vspace{0.5cm}
\begin{figure}[H]
\centering
  \includegraphics[width=5.85cm]{figuras/usmLogo.png}
\end{figure}
\vspace{0.5cm}

%\linespread{1}
\renewcommand{\baselinestretch}{1}
\hangindent=0cm
\textbf{\Large ``Sistema de monitoreo de pacientes cardíacos en tiempo real, utilizando una aplicación Android con tecnologías Bluetooth y WebSocket''\\}
\vspace{3cm}
\hangindent=0cm\large \textbf{Patricio Rodríguez Gatica}\\
\vspace{0.5cm}
\hangindent=0cm\normalsize \textbf{MEMORIA DE TITULACIÓN PARA OPTAR AL TÍTULO DE INGENIERO CIVIL TELEMÁTICO}\\
\vspace{1cm}
\hangindent=0cm\normalsize \textbf{PROFESOR GUIA: \hspace{2cm} Marcos Zúñiga B.}\\
\vspace{0.5cm}
\hangindent=0cm\normalsize \textbf{PROFESOR CORREFERENTE: \hspace{2cm} Francisco Cabezas B.}\\
\vspace{0.5cm}
\hangindent=0cm\normalsize \textbf{PROFESOR CORREFERENTE: \hspace{2cm} Daniel Erraz L.}\\

\end{center}

\thispagestyle{empty}
\newpage
\pagestyle{fancy}
\renewcommand\headrulewidth{0pt}
\renewcommand{\listfigurename}{Índice de figuras}
\renewcommand{\listtablename}{Índice de tablas}

%Numeros romanos al pie de pagina, secciones sin numero.
\pagenumbering{roman}

\section*{Agradecimientos}

Agradecer es un paso fundamental en todo desarrollo humano, puesto que es de las pocas oportunidades de reflexionar sobre quienes estuvieron y están a nuestro lado en alguna etapa de nuestra vida. Me gustaría destacar que aun cuando agradecer es una vista al pasado, no existe tiempo inconexo en el corazón y los llevo siempre conmigo.

Quiero agradecer a mi familia, mi pareja, amigos, compañeros, profesores y toda persona con quien he tenido contacto en esta etapa universitaria, todos me han formado y son parte de este trabajo de una o de otra manera.

Le dedico este trabajo a quienes siempre creyeron en mí y a quienes aun lo hacen. Porque incluso teniendo un núcleo familiar distinto, el amor y la comprensión siempre estuvieron conmigo: A mi padre Omar Bernales Vega, a mis madres Toya y Mónica Gatica y mis hermanas Bárbara, Elein y Maka. Son mi orgullo y mi ejemplo a seguir.

Por último y no por ello menos importante, a la persona que soportó mis rabietas y jornadas de estrés, quien aun me acompaña y ama de forma extraordinaria, mi Valeria.

\newpage

\section*{Resumen}

El presente documento relatará la resolución de un problema real y actual en Chile, a partir de un desafío propuesto en el contexto de las Memorias Multidisciplinarias.\par
El desafío consiste en el desarrollo de un sistema con la capacidad de monitorear pacientes de forma remota, de bajo costo y con las limitantes geográficas propias de nuestro país, teniendo en mente su aplicación a nivel público del Sistema de Salud. Para esto, se analizaron las distintas opciones existentes en el mercado y se desarrolló una solución a nivel de prototipo funcional que cumpliese con las restricciones ya mencionadas.\par
Por ser un desafío resuelto de forma multidisciplinaria es importante destacar que el desarrollo en este documento estará enfocado al área informática y de telecomunicaciones asociada a la adquisición, procesamiento, almacenamiento y envío de datos.\par
El resto del equipo final está compuesto por: Sebastián Castillo actual Ingeniero en Diseño de Productos y Felipe Cordero actual Ingeniero Civil Electrónico, ambos de la misma casa de estudios UTFSM. Ambas memorias complementan la actual en el ámbito correspondiente a sus carreras, pero lógicamente compartiendo su núcleo como proyecto conjunto.
\newpage

%\section*{Abstract}

%\newpage

\section*{Glosario}

\begin{tabular}{lcp{10.5cm}}
RF &:& Radio Frecuencia\\
	
Wi-Fi &:& Proviene del termino Wireless Fidelity. Corresponde a la norma IEEE 802.11
que define los estándares de conectividad inalámbrica para transmisión de datos entre dispositivos.\\


Esquemático &:& Es una representación pictórica de un circuito electrónico. Muestra las diferentes componentes del circuito de manera simple y las conexiones de alimentación y señales entre distintos dispositivos.\\
PCB &:& Printed Circuit Board, El circuito impreso se utiliza para conectar eléctricamente a través de las pistas conductoras, y sostener mecánicamente, por medio de la base, un conjunto de componentes electrónicos.\\
Hardware &:& Partes físicas tangibles de un sistema informático.\\
Software &:& Aplicaciones o programas que funcionan en un sistema informático.\\
Firmware &:& Programa informático que establece la lógica de mas bajo nivel que controla los circuitos electrónicos.\\
Bootloader &:& Es un programa que no tiene la totalidad de las funcionalidades para operar un sistema y está diseñado para preparar todo lo que necesita el firmware para ejecutarse.\\
IIH &:& Infección Intra-Hospitalaria
\end{tabular}

%%%%%%%%%%%%%%%%%%
%\begin{tabular}{lcp{10.5cm}}
%Relé&:& Interruptor controlado por un circuito eléctrico en el que, por medio de una bobina y un electroimán, se acciona un juego de uno o %varios contactos que permiten abrir o cerrar otros circuitos eléctricos independientes.\\
%\end{tabular}
%%%%%%%%%%%%%%%%%%

%Se sigue con numeros arabes al pie de pagina
\pagenumbering{arabic}
\tableofcontents
\newpage
\listoffigures
\newpage
\listoftables

%------------------------------------%
%			Capítulos
%------------------------------------%

%Capítulo 1: Introducción
\newpage
\chapter{Introducción}\label{intro}

\section{Memorias multidisciplinaria}
La UTFSM ha manifestado, a través de sus planes de desarrollo y ejes estratégicos, la importancia de la formación de los estudiantes en competencias transversales, el fomento de la innovación, el emprendimiento y la vinculación con la industria. Es por esto que surge en la UTFSM el proyecto de Memorias Multidisciplinarias que propone impulsar el desarrollo de una nueva industria tecnológica a través de un programa de formación para la creación sistemática y sustentable de productos de innovación y emprendimientos ligados a tecnología.

Este proyecto de Memorias Multidisciplinarias se desarrolla a través de la proposición de un desafío el cual fue otorgado por el subgerente comercial de la empresa Sistemas Expertos, José Luis Araya. Sistemas Expertos e Ingeniería de Software (SEIS) es una empresa especialista con 10 años de experiencia en el desarrollo e implementación de soluciones tecnológicas para el área de la salud.

El desafío propuesto consiste en ¿Cómo podemos incorporar a bajo costo telemedicina a la salud pública, considerando restricciones económicas y geográficas? Para esto hubo una conformación de un equipo multidisciplinario quienes desarrollaron durante un año, un plan de negocio, pruebas de concepto y prototipado de la solución con lo cual se pretende formar un emprendimiento. 


%\newpage
%A raíz de las necesidades del desafío propuesto por la empresa Sistemas Expertos, se hace necesaria la incorporación de conocimientos en el ámbito técnico a nivel Hardware, Software, Telecomunicaciones, administración de proyectos, marketing, análisis de consumidor, prototipado y posterior encapsulamiento de la solución. Es por lo anterior que el equipo está compuesto por cuatro integrantes.

\newpage
\section{Equipo}

\subsection{Felipe Cordero}
Estudiante de último año de la carrera Ingeniería Civil Electrónica con Mención en Computadores. Ha trabajado en empresas de desarrollo de hardware embebido, tiene un gran interés por crear un emprendimiento y seguir el camino de desarrollo de hardware y software. Su interés en el desafío radica en participar de un proyecto que posee todas las fases de desarrollo de hardware con un cliente desde cero. Al estar relacionado con el área de salud y conectividad  permite aportar directamente a mejorar el sistema de salud pública en Chile. 

\subsection{Vanessa Muñoz}
Estudiante 5to año de Ingeniería Comercial, 25 años. Colaborado en actividades dentro de la universidad como Preusm y actualmente trabajando por tercer año en la Feria de Empresas y Trabajo USM desempeñándose como Coordinadora General. La principal motivación por escoger este desafío es poder intervenir y mejorar algún área del sistema de la salud pública Chilena, dado que se ha podido presenciar la ineficiencia del servicio en distintas ocasiones. 
Decide abandonar el grupo por no cumplir los objetivos buscados para su trabajo de tesis.

\newpage
\subsection{Patricio Rodríguez}
Estudiante de último año en la carrera de Ingeniería Civil Telemática. Ha contribuido en distintos proyectos relacionados a procesamiento de imagen, análisis de redes, simulación, programación, entre otros. Se destaca por su gran motivación y tenacidad a la hora de desempeñar sus tareas, aportando al trabajo en equipo y facilitando la resolución de tareas. Su interés en el desafío recae en la necesidad de conectividad que este conlleva, además de estar ligado al área de la Salud. Área de especial interés considerando la distancia profesional que se puede alcanzar estudiando una carrera de Ingeniería.

\subsection{Sebastián Castillo}
Estudiante de último año en la carrera de Ingeniería en Diseño de Productos. Participado en actividades relacionadas al voluntariado, desarrollo de proyectos tecnológicos y conservación de la naturaleza. Se perfila como un profesional versátil, comprometido y que considera el trabajo multidisciplinario como fundamental en el desarrollo de soluciones para el mundo actual. El interés en este proyecto se debe a la posibilidad de poder impactar positivamente en la vida de gente con necesidades reales y mejorar, en cierta medida, su calidad de vida a través de la ingeniería, que muchas veces olvida el rol social que puede ejercer

\newpage
\section{Desafío}
La empresa Sistemas Expertos ha planteado el desafío: ¿Cómo podemos incorporar a bajo costo telemedicina a la salud pública, considerando restricciones económicas y geográficas?. 
En donde se da cuenta de la necesidad actual de aplicar las tecnologías existentes en el ámbito de salud, permitiendo de esta forma mejorar la atención. Para conseguir este objetivo se espera el desarrollo de un dispositivo electrónico con capacidad de toma de datos y envío de los mismos. Así, se pueden identificar distintas aristas a considerar, como lo son: Tipo de enfermedades y pacientes a cubrir, tipo de sensores a emplear, tipo de tecnología de comunicación, nivel de interacción con el usuario, entre otros.

Con esto en mente, se debe tomar una decisión con respecto a las enfermedades a medir ya que esto está ligado íntimamente a los sensores a utilizar pudiéndose encontrar entre ellos: electrocardiograma, saturometro, medidor de presión, termómetro, entre otros.

Además de lo anterior, para realizar la comunicación de estos datos de forma remota se contemplan distintas alternativas, entre las que se considera utilizar la infraestructura ya presente e implementada en el país, como lo son las antenas celulares conectadas directamente con el dispositivo y también utilizar conexión a internet con un intermediario como un smartphone mediante una conexión bluetooth.

Por último, respecto al nivel de interacción con el usuario, la empresa ha dejado expresa su necesidad de simplicidad en este desarrollo, descartando cualquier interfaz o comunicación directa entre el usuario final y el dispositivo. Si bien dependiendo de la tecnología a emplear esta sugerencia puede cambiar, en una primera instancia se mantiene esta línea de pensamiento en torno al desarrollo completo, intentando así mantener la sencillez en las distintas partes del dispositivo. Permitiendo de este modo reducir los datos a manipular, las interfaces a desarrollar y el riesgo de un mal uso por parte de los usuarios.

%Capítulo 2: Estado del Arte
\newpage
\input{./Capitulos/arte}

%Capítulo 3: Arquitectura de la solución
\newpage
\chapter{Arquitectura de la solución}\label{arquitectura}
El ECG (Electrocardiograma) detecta señales del cuerpo gracias a electrodos que se colocan en la superficie del cuerpo, usualmente acompañados de un gel conductor que elimina interferencia de los músculos, fuente de alimentación,  ruido externo, etc. Para que se obtenga una señal de ECG sin distorsiones excesivas, es necesario diseñar filtros que eliminen interferencias antes de analizar la información. \\

\section{DFRobot Heart Rate Monitor}\label{reqfuncional1asc}
El monitor de actividad cardíaca de la empresa DFRobot consiste en una placa que consta de un chip AD8232 en su PCB, el cual provee una clara señal de los intervalos PR y QT (Ver Figura \ref{onda})  de un electrocardiograma. Este entrega un valor análogo que puede ser leído por Arduino y con un conversor análogo-digital interpretarse en forma de gráfico.\\

\begin{figure}[H]
\centering
\includegraphics[scale=0.2]{figuras/ecg/senal.png}
\caption{Forma de onda teórica ECG}
\label{onda}
\end{figure}

Se escogió esta placa de desarrollo por disponibilidad de hardware, ya que es la única opción disponible de bajo costo, y que no necesita comprarse fuera del país. Por lo que se tomó la decisión de estudiar su funcionamiento, replicarlo para el diseño del dispositivo final y evaluar posibles mejoras. \\

\section{Biopac MP150 ECG100C}
Se tuvo la oportunidad de usar el ECG de la empresa Biopac ECG100C, el cual es de mucho mayor costo y tamaño utilizado generalmente para investigación. La idea era ver la forma de onda y comparar las frecuencias de corte utilizadas. La figura \ref{biopac} muestra el aparato encargado de tomar las señales de los electrodos y la figura \ref{frecuencias} muestra un acercamiento con las posibles frecuencias de corte configurables para los filtros.\\

\begin{figure}[H]
\centering
\includegraphics[scale=0.7]{figuras/ecg/biopac.jpg}
\caption{Biopac MP150}
\label{biopac}
\end{figure}

\begin{figure}[H]
\centering
\includegraphics[scale=0.7]{figuras/ecg/biopacecg.png}
\caption{Frecuencias de corte disponibles en Biopac ECG100C}
\label{frecuencias}
\end{figure}

Cabe destacar el alto precio de este equipo que es de 45.000 USD y grandes dimensiones ya que está diseñado para investigación lo cual servirá para saber donde tiene que apuntar la forma de onda de un ECG de grado médico.\\

\newpage
\section{AD8232}
El chip AD8232 es un integrado que permite medir las señales de los electrodos del ECG mediante amplificadores operacionales como se muestra en la figura \ref{ad8232}.

\begin{figure}[H]
\centering
\includegraphics[scale=0.4]{figuras/ecg/ad8232.png}
\caption{Diagrama de bloques funcional}
\label{ad8232}
\end{figure}

Esta figura representa el funcionamiento del integrado y el proceso que realiza en su interior desde el punto de vista de los pines. 
A partir de este diagrama y la herramienta que provee el fabricante para el diseño de filtros, se puede llegar a una configuración del integrado con componentes pasivas para el diseño del circuito impreso como se muestra en la figura \ref{ecgdesign}.\\

\begin{figure}[H]
\centering
\includegraphics[scale=0.6]{figuras/ecg/ecgdesign.png}
\caption{Herramienta para diseño de filtros}
\label{ecgdesign}
\end{figure}

Como se puede observar en la figura \ref{ecgdesign}, en el lado izquierdo se diseña un filtro pasa alto de segundo orden cuya frecuencia de corte es de 4.82[Hz]. A la derecha se muestra un filtro pasabajo sallen-key con frecuencia de corte 41.09[Hz] donde finalmente se tiene una salida hacia el Arduino. 
Estos valores se obtuvieron al ver el circuito de la PCB de DFRobot (Figura \ref{esquematico11}) y utilizando la herramienta de diseño de filtros. En caso de que se quisieran cambiar las frecuencias de corte se podría hacer un nuevo diseño utilizando el mismo programa pero cambiando y apuntando a frecuencias de corte entre 1-35[Hz] dadas por el dispositivo Biopac.

\begin{figure}[H]
\centering
\includegraphics[scale=0.45]{figuras/ecg/esquematico.png}
\caption{Esquemático ECG DFRobot}
\label{esquematico11}
\end{figure}

\newpage
\section{Comparación entre Biopac y DFRobot}
Finalmente se pudo contrastar ambas señales, se puede observar en la figura \ref{graficos} la señal de ECG del Biopac y DFRobot colocando los electrodos muy cerca para obtener las señales lo más parecidas posibles y poder compararlas. 

\begin{figure}[H]
\centering
\includegraphics[scale=0.45]{figuras/ecg/biovsdf.jpg}
\caption{Gráfico Biopac ECG100C vs ECG DFRobot}
\label{graficos}
\end{figure}

En el gráfico superior se puede observar la forma de onda del ECG AD8232 en la cual se puede observar notoriamente un ruido de la fuente. En el gráfico inferior se muestra la señal del ECG100C de Biopac, la cual es una señal muy definida a lo cual debería apuntar el diseño del ECG.
Se puede concluir de estas imágenes que la forma de onda de ambas señales son parecidas, considerando que el Biopac está diseñado para mostrar la onda completa, en cambio el AD8232 se encarga de mostrar la onda PR y QT. 
Finalmente se considera que la forma de onda que se tiene sirve para hacer mediciones eliminando el ruido de la fuente ya que es representativa por los peaks que presenta. 

\section{Eliminando ruido de la fuente}
Observando la señal que se tiene actualmente en el dispositivo mediante bluetooth en la aplicación móvil se obtiene una señal como se observa en la figura \ref{ecgfeo}.

\begin{figure}[H]
\centering
\includegraphics[scale=0.3]{figuras/ecg/ecgappmalo.jpg}
\caption{ECG en aplicación móvil sin filtro de salida}
\label{ecgfeo}
\end{figure}

Se puede notar en esta imagen que la forma de onda se mantiene al enviarla por bluetooth a la aplicación pero sigue con mucho ruido. Utilizando la herramienta de diseño de filtros para el AD8232 existe una opción para filtrar la salida de la señal final, para esto se considera utilizar un filtro pasabajo para eliminar el ruido como se muestra en la figura \ref{filtroout}.

\begin{figure}[H]
\centering
\includegraphics[scale=0.75]{figuras/ecg/filtroout.png}
\caption{Filtgro de salida R8 y C8}
\label{filtroout}
\end{figure}

Se diseñó con frecuencia de corte de $40.19[Hz]$ como se muestra en la figura \ref{filtroout} dando valores $R8 = 330[k\Omega]$ y $C8 = 12[nF]$ para no perder información de la señal y con esto limpiar solamente la señal como se muestra en el resultado de la figura \ref{ecgbonito}

\begin{figure}[H]
\centering
\includegraphics[scale=0.25]{figuras/ecg/ecgappbueno.jpg}
\caption{Comparación ECG en aplicación móvil con filtros de salida}
\label{ecgbonito}
\end{figure}

Se nota un gran cambio en la forma de la señal, donde se notan de mejor manera, sin ruido de la fuente y sin perder la forma de onda. Esta será la configuración que se utilizará para el diseño final del dispositivo.



%Capítulo 4: Alternativas de desarrollo
\newpage
\input{./Capitulos/alternativas}

%Capítulo 5: Sistema de telecomunicaciones
\newpage
\input{./Capitulos/comunicacion}

%Capítulo 6: Implementación de la solución de lado del servidor
\newpage
\input{./Capitulos/servidor}

%Capítulo 7: Implementación de la solución de lado del cliente Android
\newpage
\input{./Capitulos/servicios}

%Capítulo 8: Configuración RN4020
\newpage
\chapter{Configuración RN4020}\label{rn4020}
%\newpage
\includepdf[pages={-},scale=0.9,pagecommand={}]{figuras/anexos/esquematic.pdf}
\includepdf[pages={-},scale=1.1,pagecommand={}]{figuras/anexos/board.pdf}
\includegraphics[width=1\textwidth]{figuras/anexos/code1.jpg}
\includegraphics[width=1\textwidth]{figuras/anexos/code2.jpg}
\includegraphics[width=1\textwidth]{figuras/anexos/code3.jpg}
\includegraphics[width=1\textwidth]{figuras/anexos/code4.jpg}
\includegraphics[width=1\textwidth]{figuras/anexos/code5.jpg}
\includegraphics[width=1\textwidth]{figuras/anexos/code6.jpg}
\includegraphics[width=1\textwidth]{figuras/anexos/code7.jpg}
\includegraphics[width=1\textwidth]{figuras/anexos/code8.jpg}
\includegraphics[width=1\textwidth]{figuras/anexos/code9.jpg}
\centering
\includegraphics[width=1\textwidth]{figuras/anexos/code10.jpg}


%Capítulo 9: Integración de las componentes de la solución
\newpage
\input{./Capitulos/mldp}

%Capítulo 10: Prototipo funcional V1
\newpage
\input{./Capitulos/proto1}

%Capítulo 11: Prototipo funcional V2
\newpage
\input{./Capitulos/proto2}

%Capítulo 12: Prototipo final
\newpage
\input{./Capitulos/protof}

%Capítulo 13: Discusión
\newpage
\input{./Capitulos/discusion}

%Capítulo 14: Conclusiones
\newpage
\chapter{Conclusión}\label{conclusion}

%Capítulo 15: Anexos
\newpage
\input{./Capitulos/anexos}
		
\end{document}