%------------------------------------%
%			Inicio Preámbulo
%------------------------------------%
\documentclass[12pt,spanish]{thesis}
% Mayor compatibilidad y preferencia personal
%\usepackage[latin1]{inputenc}
\usepackage[utf8]{inputenc}
\usepackage[spanish]{babel}
\usepackage{amsmath}

% Interlineado
\usepackage{setspace}
\setstretch{1.5}

% Paquetes
\usepackage{textcomp}
\usepackage{times}
\usepackage{amssymb}
\usepackage{float}
\usepackage{color}
\usepackage{graphicx}
\usepackage{eso-pic}
\usepackage{multicol}
\usepackage{enumerate}
\usepackage{url}
\usepackage{soul}
\usepackage{fancyhdr}
\usepackage{lscape}
\usepackage{pdfpages}
\usepackage{hyperref}
\usepackage{listings}


% Márgenes
\usepackage[top=3cm,bottom=3cm,left=4.2cm,right=3cm]{geometry}
\pagestyle{empty} 
\frenchspacing
\fancyhead[L]{}
\fancyhead[C]{}
\fancyhead[R]{}
\fancyfoot[R]{\thepage}
\fancyfoot[C]{}

%------------------------------------%
%			Fin Preámbulo
%------------------------------------%


\begin{document}
\thispagestyle{empty}

\begin{center}
%\linespread{1.15}
\renewcommand{\baselinestretch}{1.15}
\textbf{\large{UNIVERSIDAD TÉCNICA FEDERICO SANTA MARÍA\\}
\normalsize{DEPARTAMENTO DE ELECTRÓNICA\\VALPARAÍSO - CHILE\\}}

\vspace{0.5cm}
\begin{figure}[H]
\centering
  \includegraphics[width=5.85cm]{figuras/usmLogo.png}
\end{figure}
\vspace{0.5cm}

%\linespread{1}
\renewcommand{\baselinestretch}{1}
\hangindent=0cm
\textbf{\Large ``Sistema de monitoreo de pacientes cardíacos en tiempo real, utilizando una aplicación Android con tecnologías Bluetooth y WebSocket''\\}
\vspace{3cm}
\hangindent=0cm\large \textbf{Patricio Rodríguez Gatica}\\
\vspace{0.5cm}
\hangindent=0cm\normalsize \textbf{MEMORIA DE TITULACIÓN PARA OPTAR AL TÍTULO DE INGENIERO CIVIL TELEMÁTICO}\\
\vspace{1cm}
\hangindent=0cm\normalsize \textbf{PROFESOR GUIA: \hspace{2cm} Marcos Zúñiga B.}\\
\vspace{0.5cm}
\hangindent=0cm\normalsize \textbf{PROFESOR CORREFERENTE: \hspace{2cm} Francisco Cabezas B.}\\
\vspace{0.5cm}
\hangindent=0cm\normalsize \textbf{PROFESOR CORREFERENTE: \hspace{2cm} Daniel Erraz L.}\\

\end{center}

\thispagestyle{empty}
\newpage
\pagestyle{fancy}
\renewcommand\headrulewidth{0pt}
\renewcommand{\listfigurename}{Índice de figuras}
\renewcommand{\listtablename}{Índice de tablas}

%Numeros romanos al pie de pagina, secciones sin numero.
\pagenumbering{roman}

\section*{Agradecimientos}

Agradecer es un paso fundamental en todo desarrollo humano, puesto que es de las pocas oportunidades de reflexionar sobre quienes estuvieron y están a nuestro lado en alguna etapa de nuestra vida. Me gustaría destacar que aún cuando agradecer es una vista al pasado, no existe tiempo inconexo en el corazón y los llevo siempre conmigo.

Quiero agradecer a mi familia, mi pareja, amigos, compañeros, profesores y toda persona con quien he tenido contacto en esta etapa universitaria, todos me han formado y son parte de este trabajo de una o de otra manera.

Le dedico este trabajo a quienes siempre creyeron en mí y a quienes aún lo hacen. Porque incluso teniendo un núcleo familiar distinto, el amor y la comprensión siempre estuvieron conmigo: A mi padre Omar Bernales Vega, a mis madres Toya y Mónica Gatica y mis hermanas Bárbara, Elein y Maka. Son mi orgullo y mi ejemplo a seguir.

Por último y no por ello menos importante, a la persona que soportó mis rabietas y jornadas de estrés, quien aún me acompaña y ama de forma extraordinaria, mi Valeria.

\newpage

\section*{Resumen}

El presente documento relatará la resolución de un problema real y actual en Chile, a partir de un desafío propuesto en el contexto de las Memorias Multidisciplinarias.\par
El desafío consiste en el desarrollo de un sistema con la capacidad de monitorear pacientes de forma remota, de bajo costo y con las limitantes geográficas propias de nuestro país, teniendo en mente su aplicación a nivel público del Sistema de Salud. Para esto, se analizaron las distintas opciones existentes en el mercado y se desarrolló una solución a nivel de prototipo funcional que cumpliese con las restricciones ya mencionadas.\par
Por ser un desafío resuelto de forma multidisciplinaria es importante destacar que el desarrollo en este documento estará enfocado al área informática y de telecomunicaciones asociada a la adquisición, procesamiento, almacenamiento y envío de datos.\par
El resto del equipo final está compuesto por: Sebastián Castillo actual Ingeniero en Diseño de Productos y Felipe Cordero actual Ingeniero Civil Electrónico, ambos de la misma casa de estudios UTFSM. Ambas memorias complementan la actual en el ámbito correspondiente a sus carreras, pero lógicamente compartiendo su núcleo como proyecto conjunto.
\newpage

%\section*{Abstract}

%\newpage

\section*{Glosario}

\begin{tabular}{lcp{10.5cm}}

2G &:& Segunda generación de telefonía móvil.\\	
	
RF &:& Radio Frecuencia.\\
	
Wi-Fi &:& Proviene del termino Wireless Fidelity. Corresponde a la norma IEEE 802.11
que define los estándares de conectividad inalámbrica para transmisión de datos entre dispositivos.\\

IDE &:& De sus siglas en inglés: Entorno de desarrollo integrado.\\

SO &:& Sistema Operativo.\\

Framework &:& Estructura conceptual y tecnológica de soporte definida, normalmente con artefactos o módulos de software concretos, en base a la cual otro proyecto de software puede ser organizado y desarrollado. Típicamente, puede incluir soporte de programas, librerias y un lenguaje interpretado entre otros programas para ayudar a desarrollar y unir los diferentes componentes.\\

SCRUM &:& Metodología de desarrollo ágil caracterizado principalmente por: Adaptabilidad a cambios, solapamiento de fases de desarrollo y foco en resultados incrementales.\\

Sprint &:& Iteración recurrente utilizada en la metodología ágil SCRUM, corresponde a bloques temporales cortos y fijos.\\

UART &:& De sus siglas en inglés: Transmisor-Receptor Asíncrono Universal, es el dispositivo que controla los puertos y dispositivos serie.\\

UUID &:& Identificador único universal o universally unique identifier (UUID) es un número de 16 bytes (128 bits - Ejemplo: 550e8400-e29b-41d4-a716-446655440000).\\

RFCOMM &:& De sus siglas en inglés: Comunicación por radio frecuencia, es un conjunto simple de protocolos de transporte, construido sobre el protocolo L2CAP. El protocolo está basado en el estándar ETSI TS 07.10.\\

\end{tabular}

\newpage
\begin{tabular}{lcp{10.5cm}}
	
Backend &:& Capa posterior de un servicio informático, la cual no está pensada para el usuario y tiene como fin sustentar el servicio como proveer datos o lógica.\\
IP &:& Una dirección IP es un número que identifica, de manera lógica y jerárquica, a una Interfaz en red. La más común es la IPv4 aunque también existe la IPv6 y su única diferencia es la cantidad de bits empleados: 32 y 128 bits respectivamente.\\

Hosting &:& Servicio de alojamiento en la web.\\

HTTP &:& Protocolo de transferencia de hipertexto, es el protocolo de comunicación que permite las transferencias de información en la World Wide Web.\\

Instancia &:& En el contexto informático entiéndase como la "materialización" de cierta abstracción como lo puede ser un modelo o esquema.\\

Puerto &:& Número de 16 bits, usado por el protocolo host-a-host para identificar a qué protocolo de más alto nivel o programa de aplicación (proceso) debe entregar los mensajes de entrada.\\

Dominio &:& Nombre que sirve de identificador para una dirección IP. \\

DNS &:& El sistema de nombres de dominio (DNS, por sus siglas en inglés, Domain Name System) es un sistema de nomenclatura jerárquico descentralizado para dispositivos conectados a redes IP como Internet o una red privada. Este sistema asocia información variada con nombre de dominio asignado a cada uno de los participantes.\\


Proxy &:& Agente o sustituto autorizado para actuar en nombre de otra máquina o sistema.\\

TCP &:& Protocolo de control de transmisión, es uno de los protocolos fundamentales en Internet.\\

\end{tabular}

\newpage
\begin{tabular}{lcp{10.5cm}}
Socket &:& Concepto abstracto por el cual dos programas (posiblemente situados en computadoras distintas) pueden intercambiar cualquier flujo de datos, generalmente de manera fiable y ordenada..\\

WebSocket &:& Tecnología que proporciona un canal de comunicación bidireccional y full-duplex sobre un único socket TCP.\\

Baudio &:& Unidad de medida de la velocidad de transmisión de señales que se expresa en símbolos por segundo.\\

\end{tabular}
%%%%%%%%%%%%%%%%%%
%\begin{tabular}{lcp{10.5cm}}
%Relé&:& Interruptor controlado por un circuito eléctrico en el que, por medio de una bobina y un electroimán, se acciona un juego de uno o %varios contactos que permiten abrir o cerrar otros circuitos eléctricos independientes.\\
%\end{tabular}
%%%%%%%%%%%%%%%%%%

%Se sigue con numeros arabes al pie de pagina
\pagenumbering{arabic}
\tableofcontents
\newpage
\listoffigures
\newpage
\listoftables

%------------------------------------%
%			Capítulos
%------------------------------------%

%Capítulo 1: Introducción
\newpage
\input{./Capitulos/1_intro}

%Capítulo 2: Estado del Arte
\newpage
\chapter{Estado del arte}\label{arte}
En el marco del desarrollo del desafío de Sistemas Expertos, se planteó generar un dispositivo de monitoreo a distancia de pacientes. Para esto se comenzó a estudiar aspectos relacionados con la Telemedicina y sus implicancias en el avance del monitoreo Remoto de Paciente (RPM, por sus siglas en inglés). La Telemedicina es, en principio, la tecnología que permite entregar cuidados médicos a través de la infraestructura de las telecomunicaciones, permitiendo a los médicos diagnosticar o evaluar enfermedades sin la necesidad de un control presencial.
Para poder comprender en qué se encuentra la realidad nacional y latinoamericana es de suma importancia revisar algunos casos dónde se apliquen dispositivos de telemedicina bajo la modalidad de monitorear y digitalizar la información, considerando que el objetivo del proyecto se limita a esas dos acciones.

\section{ViSi Mobile\textregistered}
ViSi Mobile\textregistered\ \cite{visi}, si bien se utiliza en el cuerpo, es una estación que procesa los datos de otros sensores que van colocados en el cuerpo y que a su vez se conectan al módulo central de procesamiento como se puede observar en la imagen 1 lo que es necesario categorizarlo como un producto modular. Los sensores se encargan de medir pulso, respiración, SpO2, presión sanguínea continua no invasiva y temperatura de la piel. El principal objetivo es permitir monitorear al paciente de forma continua dentro del hospital, sin intervenir de manera negativa en el flujo de trabajo que allí existe (ViSi Mobile\textregistered\ System, s. f.). ViSi Mobile\textregistered\ se encarga de recopilar los datos que cada sensor pueda otorgar para luego enviarlos de manera simultánea a un smartphone, una plataforma online de monitoreo y además directo a la estación de trabajo del médico a cargo, permitiendo así una atención eficiente. Esto lo logra haciendo uso de una red existente de Wi-Fi y encriptación WPA2 para la seguridad en la comunicación\cite{visi_tel}.

\begin{figure}[H]
	\centering
	\includegraphics[scale=0.3]{figuras/estadoarte/visi/visi.jpg}
	\caption{Interfaz de usuario ViSi Mobile\textregistered}
	\label{visi1}
\end{figure}

Se puede observar en la figura \ref{visi1} la interfaz que puede ver el paciente al utilizar el dispositivo.

\begin{figure}[H]
	\centering
	\includegraphics[scale=0.7]{figuras/estadoarte/visi/wear.jpg}
	\caption{Modo de uso ViSi Mobile\textregistered}
	\label{visi2}
\end{figure}

En la imagen de la figura \ref{visi2} se puede observar los sensores conectados al cuerpo que convergen al dispositivo que toma las señales.

\section{Qardiocore\textregistered}

QardioCore\textregistered\ \cite{qardio} es un monitor de electrocardiograma inalámbrico diseñado para mejorar la detección y manejo de las condiciones cardíacas. Seis sensores se encargan de grabar y analizar sobre 20 millones de puntos de datos durante todo el día junto con otros signos vitales. Este dispositivo está orientado a personas con alto nivel de riesgo cardíaco causado por predisposición familiar, historial de ataques al corazón, presión alta, colesterol alto, diabetes o exceso de peso. Monitorea de forma precisa y continua la salud del corazón. El dispositivo graba datos de ECG, pulso, variación de pulso, temperatura corporal, ritmo respiratorio y niveles de estrés. A diferencia de los ECG tradicionales, QardioCore no utiliza gel ni cables para monitorear y funciona entre -20ºC y 60ºC. Adicionalmente es resistente al agua y su batería dura alrededor de un día. Respecto a las especificaciones técnicas, es capaz de funcionar con una frecuencia de 600 muestras por segundo y una resolución de 16 bit, apoyándose en comunicación Bluetooth 4.0 y plataforma exclusiva iOS (9.0 o superior)\cite{qardio_tel}.

\begin{figure}[H]
	\centering
	\includegraphics[scale=0.5]{figuras/estadoarte/qardio/qardio.jpg}
	\caption{Qardiocore\textregistered\ multisensor}
	\label{qardio1}
\end{figure}

Se puede observar el dispositivo Qardiocore en la figura \ref{qardio1} que se conecta a un smartphone para mostrar los datos que se están tomando.

\newpage
Además como se muestra en la figura \ref{qardio2} es de simple uso, funciona como un cinturón en el pecho del paciente.

\begin{figure}[H]
	\centering
	\includegraphics[scale=0.5]{figuras/estadoarte/qardio/wear.png}
	\caption{Modo de uso Qardiocore}
	\label{qardio2}
\end{figure}

\section{Nuubo\textregistered}

Nuubo\textregistered\ \cite{nuubo} proporciona una nueva perspectiva en la monitorización cardiológica remota e inalámbrica. La plataforma de Nuubo\textregistered\, nECG platform, permite la captura del ECG dinámico a través de un innovador sistema que está basado en textiles biomédicos de nueva generación, y es rentable, remoto, continuo y no invasivo. Además, puede ser utilizado simultáneamente con uno o varios pacientes. Se basa en tecnología Bluetooth v2.0 + EDR (PC y móvil), con una frecuencia de 250 muestras por segundo y 12 bit de resolución\cite{nuubo_tel}.
La tecnología de electrodos textiles desarrollada por Nuubo\textregistered\ simplifica enormemente los incómodos procedimientos tradicionales de conexión de electrodos, reduciéndolos al sencillo acto de vestir la camiseta nECG SHIRT que se muestra en la figura \ref{shirt}.

\begin{figure}[H]
	\centering
	\includegraphics[scale=0.6]{figuras/estadoarte/nuubo/shirt.png}
	\caption{nECG Shirt}
	\label{shirt}
\end{figure}

El tejido elástico se adapta a los movimientos del paciente, quien puede realizar su actividad física diaria sin estar limitado por cables y sin necesidad de depender de personal médico especializado. Estas características junto con la información de contexto, la actividad física del paciente y su posición/postura, permite el desarrollo de un nuevo rango de soluciones y casos de uso. 

\begin{figure}[H]
	\centering
	\includegraphics[scale=0.7]{figuras/estadoarte/nuubo/nuubo.png}
	\caption{Sistema Nuubo}
	\label{nuubo}
\end{figure}

Como se puede observar en la figura \ref{nuubo} la polera toma los datos que son enviados a un dispositivo movil o un computador para que sea visto por el doctor de manera remota.

%Capítulo 3: Arquitectura de la solución
\newpage
\chapter{Arquitectura de la solución}\label{arquitectura}

Luego de analizar las necesidades del proyecto y el estado actual de la industria frente al desafío, el siguiente paso es establecer una arquitectura base con la cual definir las partes más relevantes del sistema. En la figura \ref{arqui} se pueden apreciar las 3 secciones relevantes y detalladas en las siguientes secciones.

\begin{figure}[H]
	\centering
	\includegraphics[scale=0.43]{figuras/arquitectura/arqui.png}
	\caption{Arquitectura referente}
	\label{arqui}
\end{figure}

\newpage
\section{Adquisición de datos}
Una primera necesidad del proyecto es la obtención y procesamiento de los datos, en una primera instancia se omitirá la definición del almacenamiento, permitiendo enfocar los esfuerzos en la selección de sensores, su interconexión y la plataforma que sustente su funcionamiento. Algunos de los requisitos en este apartado son:

\begin{enumerate}
	\item \textbf{Variedad:}
	Considerando la gama de enfermedades que se podrían cubrir, es importante contemplar una plataforma que permita trabajar con gran cantidad sensores.
	\item \textbf{Comodidad:}
	A raíz de que este apartado es el único que tendrá contacto con el paciente es importante pensar en el confort ofrecido, descartando opciones que afecten este apartado, como placas demasiado grandes o pesadas.
	\item \textbf{Flexibilidad:}
	Al estar en un proceso iterativo en búsqueda de opciones, un factor a considerar es la flexibilidad que nos puedan ofrecer las distintas opciones, permitiéndonos realizar cambios importantes sin afectar en gran medida las decisiones ya tomadas.
\end{enumerate}

\newpage
\section{Comunicación}
En el apartado 

\newpage
\section{Servicio web}



%Capítulo 4: Alternativas de desarrollo
\newpage
\chapter{Alternativas de desarrollo}\label{alternativas}
En el presente capítulo se ahondará en las distintas alternativas de diseño que existen para el prototipo con los distintos sensores requeridos, además de establecer la comunicación y el envío de la información tomada del paciente. Las etapas para el desarrollo del prototipo constan de: Elección de sistema de procesamiento o unidad central, sensores a utilizar y forma de comunicación inalámbrica.
\section{Plataforma de desarrollo}\label{proce}
Al fabricar un prototipo, el desarrollador debe construir el hardware sobre el cual correrá el software del producto que ha diseñado, por lo que debe tomar componentes de diversos proveedores, integrarlos y hacerlos funcionar como un conjunto. Por esa razón se popularizó el uso de plataformas de desarrollo electrónico.  \\
Por lo general, estas son placas que integran microcontroladores, circuitos y componentes electrónicos que le proporcionan diversas capacidades básicas y a partir de esto se puede evaluar la compatibilidad del diseño tanto en hardware como en software antes de enviar a fabricar el producto final. 

\newpage
\subsection{Arduino}
Arduino es una plataforma de desarrollo de bajo costo que permite crear proyectos de base tecnológica de forma sencilla y barata, que consta de entradas análogas, entradas y salidas digitales, PWM, comunicación serial, etc. 

Uno de los beneficios de Arduino es que provee módulos de desarrollo de bajo costo para trabajar con integrados y estudiar su funcionamiento y prototipado.
Arduino trabaja con una gran variedad microcontroladores AVR que diferencia por modelos dependiendo de las necesidades de proyecto, motivo por el cual varía en precio. 

En primera instancia se puede trabajar con un modelo Arduino UNO que es de bajo costo y permite leer señales análogas y traducirlas en su conversor análogo-digital y dependiendo de las necesidades se puede conseguir otro modelo como Arduino Mega que ofrece mayores prestaciones. 
\subsection{Raspberry}
Raspberry es una computadora de placa reducida (SBC por sus siglas en inglés) de bajo costo, con el objetivo de estimular la enseñanza de ciencias de la computación, no obstante, es de propiedad registrada para poder mantener el control de la 8 plataforma y no se generan excesivas variantes como es el caso de Arduino. El software que usa es open source, aunque es capaz de ejecutar incluso una versión de Windows 10. Por lo mismo su capacidad de procesar señales es mayor y permite ejecutar proyectos más complejos. No se define si es que pueden o no ser usadas en desarrollos comerciales.

\newpage
\subsection{Beaglebone}
Beaglebone black es la última iteración de la serie Beaglebone y su versión pequeña. Esencialmente es similar a Raspberry, diferenciándose en cosas como la capacidad para iniciarse sin la necesidad de instalar ningún sistema operativo ya que tiene memoria integrada, no así Raspberry. Adicionalmente cuenta con una cantidad de entradas sustancialmente mayor, por lo que permite hasta el doble de conexiones que su competencia directa. Como si es no fuera suficiente, la arquitectura del procesador que incluye Beaglebone black permite que rinda hasta el doble de rápido que su contraparte en Raspberry pi. \\
Al igual que su competencia, Beaglebone ofrece mucho mas procesamiento que el necesario por lo que se descarta como una opción para el desarrollo inicial del prototipo, de acuerdo a las necesidades que vayan surgiendo se puede considerar nuevamente como una opción.
\newpage
\section{Sensores}
Hablando con la contraparte de Sistemas Expertos se decidió, a partir de la información que proveen ellos, que las enfermedades  mas comunes son las afecciones cardíacas y también es necesario tener un control de la temperatura de los pacientes a la hora de leer sus signos vitales. \\
Por otra parte se propuso utilizar una IMU para detectar si algún paciente sufre una caída, este con el fin de emitir una alarma para llamar una ambulancia en caso de ser necesario.
\subsection{ECG}
Electrocardiograma o ECG es el proceso de registrar la actividad eléctrica del corazón en un periodo de tiempo usando electrodos directamente en la piel.\\
Lo fundamental será buscar un circuito de desarrollo para realizar una prueba de concepto, en la cual se puedan tomar los datos y manejar.
\subsubsection{DFRobot Heart Rate Monitor Sensor}
El monitor de actividad cardiaca de la empresa DFRobot se usa para medir la actividad electrica del corazón con un integrado AD8232\cite{ad8232} que toma señales análogas de los electrodos y utiliza amplificadores para tener una mejor lectura de los datos.\\
Utilizando un Arduino es posible leer los datos tomados de los electrodos y convertirlos a información digital que puede ser enviada por comunicación serial.\\

\begin{figure}[H]
	\centering
	\includegraphics[scale=0.5]{figuras/sensor/ecg/ecg.png}
	\caption{Placa de desarrollo ECG}
	\label{ecg}
\end{figure}

Como se puede observar en la figura \ref{ecg} posee conexión simple para electrodos y salida análoga, lo que permitirá una rápida prueba de concepto para utilizar este integrado en el diseño del dispositivo final. Además este provee filtros que se van a estudiar mas adelante.


\subsubsection{ADS1298}
El integrado ADS1298 de la empresa Texas Instrument ofrece un ECG con 8 amplificadores programables de bajo ruido y 8 conversores Análogo-digital de alta resolución.\\
Utilizado para instrumentación medica y lectura tanto de ECG como EMG (Electromiograma) y EEG (Electroencefalograma).\\
El integrado ADS1298 es una buena opción para un desarrollo de ECG en el futuro de grado médico, pero es de un precio 10 veces mayor al dispositivo de DFRobot por lo que se va a descartar para el prototipo funcional.
\subsection{Temperatura}
Cuando se requiere realizar alguna medición a un paciente siempre es necesario conocer su temperatura corporal que sirve como información complementaria a los profesionales de la salud es por esto que se evaluarán termistores que permitan la lectura de este dato.
\subsubsection{Lilypad Temperature Sensor}
Dentro de la tendencia del hardware abierto, uno de los proyectos más destacados es Lilypad Arduino, un conjunto de piezas electrónicas que se pueden coser a los tejidos para darles interactividad con sensores, luces o sonidos.\\
Entre estos sensores tenemos un sensor de temperatura compuesto por un termistor MCP9700 el cual ofrece una resolución de $\pm2^\circ C.$\\

\newpage
La particularidades que ofrece este sensor es ser de muy bajo costo y a su vez es impermeable por lo que permitiría incorporarlo en el wearable de forma permanente sin dañar la componente.

\begin{figure}[H]
	\centering
	\includegraphics[scale=0.7]{figuras/sensor/t/lilypad.jpg}
	\caption{Sensor de temperatura Lilypad}
	\label{Lilypad}
\end{figure}

Como se puede observar en la figura \ref{Lilypad}, Lilypad ofrece una PCB impermeable con 3 terminales que permiten utilizar un hilo conductor para coser este a la ropa.

\subsubsection{DS18B20}
El sensor DS18B20\cite{temp} es un termómetro digital que ofrece una medida de 9 a 12 bits de resolución. Se comunica mediante el bus 1-Wire (protocolo de comunicación en serie diseñado por Dallas Semiconductor el cual está basado en un maestro y varios esclavos en una sola linea de datos) lo cual permitiría, en caso de ser necesario, incorporar mas sensores para obtener una medida con mayor precisión. 
Este termómetro digital ofrece una resolución de $\pm0.5^\circ C$ y a su vez ofrece un formato impermeable en forma cilíndrica como se observa en la figura \ref{DS18B20}.

\begin{figure}[H]
	\centering
	\includegraphics[scale=0.2]{figuras/sensor/t/ds18.jpg}
	\caption{Sensor de temperatura DS18B20}
	\label{DS18B20}
\end{figure}

\subsection{Ritmo Respiratorio}
Para medir el ritmo respiratorio, sensor que la contraparte pidió estudiar utilizando una tela conductora, se consideró el uso de la tela conductiva MedTex, la cual entrega un valor de resistividad en ohms en su estado en reposo y este varía dependiendo de su estiramiento.\\

\begin{figure}[H]
	\centering
	\includegraphics[scale=0.5]{figuras/tela/medtex.jpg}
	\caption{Tela Conductiva MedTex}
	\label{medtex}
\end{figure}

Para estudiar la factibilidad de la tela conductiva que se puede observar en la figura \ref{medtex} se cortó una tira de un tamaño $20x2 [cm]$ en estiramiento cero sobre una banda elástica que luego fue cosida como cinturón de pecho. Una vez colocada en cada extremo de la tela conductiva se colocó un caimán conectado a su vez a un multitester que permitía visualizar variaciones de la resistividad de la tela a partir de su estiramiento. 

\newpage
\begin{table}[H]
	\centering
	\begin{tabular}{| c | c | c |}
		\hline
		\multicolumn{1}{|c|}{\textbf{Resistencia en reposo [$\Omega$]}}&
		\multicolumn{1}{c|}{\textbf{Resistencia en estiramiento [$\Omega$]}}&
		\multicolumn{1}{|c|}{\textbf{\% de variación)}}\\ \hline
		$4.8$  & $4.6$  & $0.1420$  \\ \hline
		$4.7$  & $4.5$ & $0.1421$ \\ \hline
		$4.8$ & $4.7$  & $0.0722$  \\ \hline
	\end{tabular}
	\caption{Valores resistencia Tela MedTex en Pectorales}
	\label{tablatex1}
\end{table}

\begin{table}[H]
	\centering
	\begin{tabular}{| c | c | c |}
		\hline
		\multicolumn{1}{|c|}{\textbf{Resistencia en reposo [$\Omega$]}}&
		\multicolumn{1}{c|}{\textbf{Resistencia en estiramiento [$\Omega$]}}&
		\multicolumn{1}{|c|}{\textbf{\% de variación)}}\\ \hline
		$4.7$  & $4.6$  & $0.0699$  \\ \hline
		$4.7$  & $4.6$ & $0.0699$ \\ \hline
		$4.6$ & $4.5$  & $0.0723$  \\ \hline
	\end{tabular}
	\caption{Valores resistencia Tela MedTex en Plexo}
	\label{tablatex2}
\end{table}

\begin{table}[H]
	\centering
	\begin{tabular}{| c | c | c |}
		\hline
		\multicolumn{1}{|c|}{\textbf{Resistencia en reposo [$\Omega$]}}&
		\multicolumn{1}{c|}{\textbf{Resistencia en estiramiento [$\Omega$]}}&
		\multicolumn{1}{|c|}{\textbf{\% de variación)}}\\ \hline
		$4.7$  & $4.6$  & $0.0699$  \\ \hline
		$4.8$  & $4.7$ & $0.0722$ \\ \hline
		$4.7$ & $4.5$  & $0.1421$  \\ \hline
	\end{tabular}
	\caption{Valores resistencia Tela MedTex en Estomago}
	\label{tablatex3}
\end{table}

Se puede observar en las tablas \ref{tablatex1}, \ref{tablatex2} y \ref{tablatex3} las variaciones de resistencias no son constantes ni regulares, el mismo estiramiento a veces no producía la misma variaciones de resistencia. Además por mínimas variaciones en el movimiento también habían variaciones que arruinaban la medición, por lo que esta alternativa no sería viable para medir el ritmo respiratorio.

\subsection{Unidad de movimiento inercial (IMU)}
Una unidad de movimiento inercial o IMU (del inglés inertial measurement unit), es un dispositivo electrónico que mide la aceleración, inclinación y las fuerzas gravitacionales, usando una combinación de acelerómetros y giroscopios.
\subsubsection{MPU-9250}
El integrado MPU-9250 es un modulo multi-chip que consiste en 2 integrados en un empaquetado QFN. Este provee un giroscopio de 3 ejes y un acelerómetro de 3 ejes.
Este chip provee tres conversores análogo-digital de 16 bits para digitalizar las salidas del giroscopio, acelerómetro y giroscopio de manera independiente.\\
Sparkfun provee una PCB de desarrollo para realizar pruebas como se muestra en la imagen \ref{imu1}.

\begin{figure}[H]
	\centering
	\includegraphics[scale=1.5]{figuras/sensor/imu/imu1.jpg}
	\caption{IMU Sparkfun MPU-9250}
	\label{imu1}
\end{figure}

Es importante destacar la orientación indicada por el fabricante al momento de diseñar el equipo electrónico que son predefinidas como se puede ver en el caso de la figura \ref{imu1} en la cual se muestran los ejes X, Y y Z tanto para el acelerómetro como para el giroscopio. 
\newpage
Como se observa en la figura \ref{imu11} se muestra además de los ejes de aceleración también las coordenadas de navegación (roll, pitch, yaw). 

\begin{figure}[H]
	\centering
	\includegraphics[scale=0.5]{figuras/sensor/imu/imu11.jpg}
	\caption{Ejes IMU MPU-9250}
	\label{imu11}
\end{figure}

\section{Comunicación}
Para el proyecto se consideran distintas alternativas de conexión, las cuales deben seguir ciertos aspectos relevantes según las especificaciones del dispositivo a implementar. 


\begin{enumerate}
	\item \textbf{Alta disponibilidad:}
	Se requiere que la tecnología a emplear permita establecer conexiones a lo largo del tiempo, presentando pocas o de preferencia nulas desconexiones o incapacidades de conexión.
	\item \textbf{Gran cobertura:}
	Considerando la envergadura inicial del proyecto, Chile, es de vital importancia que la tecnología a utilizar permita generar conexiones en la mayor parte del territorio nacional.
	\item \textbf{Bajo costo:}
	Dentro de los requerimientos del proyecto se encuentra el desarrollo e implementación a bajo costo de la solución final, por tanto la tecnología de comunicación a emplear debe seguir esta directriz para ser seleccionada.
	\item \textbf{Baja complejidad:}
	Considerando que el dispositivo en cuestión debiera ser lo más autónomo y sencillo de configurar, es relevante considerar tecnologías de comunicación que no requieran de complejas operaciones para su uso e implementación.
	\item \textbf{Escalabilidad:}
	Si bien es un aspecto dependiente de los anteriores requerimientos, es relevante considerarlo por separado como la medida que representa la capacidad de atender una gran cantidad de conexiones (usuarios en definitiva). 
\end{enumerate}

En base a lo anterior, se hace un análisis rápido de diferentes alternativas que podrían utilizarse en el proyecto:

\begin{enumerate}
	\item \textbf{Antenas de RF:}\cite{RF}
	Comunicación generada en las bandas situadas entre los 3 kilohercios (KHz) y 300 Gigahercios (GHz). Esta tecnología incluye distintas otras tecnologías como las redes celulares, pero en este apartado se especifica el uso de bandas no utilizadas por esta y otras tecnologías. Permitiendo una conexión directa de antena a antena a una frecuencia específica a determinar.
	
	\item \textbf{Comunicación satelital:}\cite{satelite}
	Comunicación  por medio de ondas electromagnéticas transmitidas gracias a la presencia en el espacio de satélites artificiales situados en órbita alrededor de la Tierra. Dentro de esta tecnología se pueden encontrar dos grandes clasificaciones: Satélites activos y Satélites pasivos, los cuales se diferencian por la amplificación o de las señales antes de ser reenviadas a la Tierra, respectivamente.
	
	\item \textbf{Redes Wi-Fi:}\cite{wifi}
	También llamada WLAN (Wireless lan, red inalámbrica) o estándar IEEE 802.11, es una de las tecnologías de comunicación inalámbrica mediante ondas más utilizada hoy en día. Existen distintas variantes de este estándar de comunicación, entre los que se destacan 802.11g y 802.11n, por su uso actual en dispositivos comerciales.
	
	\item \textbf{Redes celulares:}\cite{celular}
	Consiste en una red de celdas cada una con su propio transmisor, conocidas como estación base. Ampliamente utilizadas en la actualidad, lográndose encontrar hasta 7 compañías distintas que ofrecen sus servicios en Chile: Movistar, Entel, WOM, Claro, Virgin, VTR, SIMPLE.
	\begin{enumerate}
		\item \textbf{Comunicación directa:}
		Tipo de comunicación en donde el dispositivo posee la capacidad de conectarse, registrarse y hacer uso completo de la infraestructura proporcionada por distintas compañías.
		
		\item \textbf{Comunicación indirecta:}
		Tipo de comunicación con la cual el dispositivo requiere de un paso intermedio de comunicación para generar la conexión a la red requerida, para este paso se puede destacar el uso de Bluetooth para la comunicación con otro dispositivo con la capacidad de conectarse de forma directa a las redes celulares.
	\end{enumerate}	
\end{enumerate}	

Luego de caracterizar las distintas tecnologías disponibles para su uso en el proyecto, se procede a analizar sus cualidades en función de las 4 especificaciones anteriores: 

\begin{table}[H]
	\centering
	\begin{tabular}{| c | c | c | c | c | c |}
		\hline
		\multicolumn{1}{|c|}{\textbf{Tecnología}}&
		\multicolumn{1}{c|}{\textbf{Disponibilidad}}&
		\multicolumn{1}{|c|}{\textbf{Cobertura}}&
		\multicolumn{1}{|c|}{\textbf{Costo}}&
		\multicolumn{1}{|c|}{\textbf{Complejidad}}&
		\multicolumn{1}{|c|}{\textbf{Escalable}}\\ \hline
		Antenas RF  & Alta  & Baja & Alta & Alta & Baja \\ \hline
		Satelital  & Alta & Alta & Alta & Alta & Baja\\ \hline
		Wi-Fi & Alta & Baja  & Bajo & Media & Alta\\ \hline
		Directa & Alta  & Alta  & Medio & Baja & Alta\\ \hline
		Indirecta & Alta  & Alta/Baja & Bajo & Media & Alta/Media\\ \hline
	\end{tabular}
	\caption{Comparativa tecnologías / requerimientos}
	\label{tablacompara_telecomunicaciones}
\end{table}

A raíz de lo anterior se destaca el uso de tecnologías con redes celulares por su lineamiento con el proyecto. La tecnología Wi-Fi se descarta por ser de baja cobertura (en una primera instancia y pensando a nivel nacional) y esto a su vez ser un ámbito crítico para el proyecto. A continuación se presentan alternativas para la plataforma Arduino en torno a las tecnologías ya mencionadas.

\subsection{Celular directa: GPRS/GSM shield}
Para integrar conexión a redes celulares en el dispositivo es necesario considerar un GPRS shield compatible con socket xbee.\\ GPRSBee cumple con los requerimientos a un precio no menor (aproximadamente 36.000 CLP). 
Se puede observar en la figura \ref{gprs} el módulo disponible para desarrollo.

\begin{figure}[H]
	\centering
	\includegraphics[scale=0.8]{figuras/com/gprs.png}
	\caption{Modulo GPRSBee}
	\label{gprs}
\end{figure}

Cabe destacar que para poder utilizar este módulo es necesario incluir una antena que se puede ver en la figura \ref{antena}

\begin{figure}[H]
	\centering
	\includegraphics[scale=0.5]{figuras/com/antena.jpg}
	\caption{Antena GPRSBee}
	\label{antena}
\end{figure}

Al considerar este módulo se puede concluir que es incompatible con el diseño del wearable ya que la antena es muy grande (aproximadamente $57.40[mm]$ ) lo que sería molesto en el dispositivo final. Otro punto en contra de este módulo es el alto costo y el consumo energía que lo hace incompatible con la autonomía que se desea.

\subsection{Celular indirecta: Bluetooth BLE shield}
Para integrar Bluetooth en el dispositivo se considera un BLEBee el cual ofrece Bluetooth versión 4.1 y comunicación UART mediante un puerto XBEE. \\
El shield Bluetooth posee un módulo RN4020\cite{RN4020} el cual ofrece una antena para la comunicación en su misma placa lo que facilita el diseño como se puede observar en la figura \ref{bt}.

\begin{figure}[H]
	\centering
	\includegraphics[scale=0.5]{figuras/com/rn4020.png}
	\caption{Bluetooth RN4020}
	\label{bt}
\end{figure}

Es importante destacar que para mejorar el diseño, el fabricante recomienda dejar expuesta la antena y se destaca la comunicación UART para las configuraciones futuras del sistema.

\newpage

\section{Conclusiones}
En esta sección, tomando en cuenta las opciones vistas en el mismo capítulo, se seleccionarán las primeras componentes a utilizar para el prototipo funcional y para realizar la prueba de concepto con lo que se va a basar el proyecto.
\subsection{Plataforma de desarrollo}
Para la plataforma de desarrollo se va a escoger trabajar con Arduino ya que este posee distintas versiones con distintos costos, los cuales son menores que Raspberry o Beaglebone. Además cabe destacar que el sistema que se quiere desarrollar es toma de datos y envío de información por lo que no se va a requerir tanto procesamiento. Arduino cubre las necesidades en su versión UNO con un microcontrolador ATMega328p, en caso de necesitar uno de mayor capacidad se puede optar por un Arduino Mega.
\subsection{Electrocardiograma}
Para el sensor de electrocardiograma se utilizará el monitor de actividad cardíaca de DFRobot, esto debido a que es la única opción que se puede conseguir en el país para no retrasar el desarrollo. Este sensor es de muy bajo costo (alrededor de 19.500 CLP en MCIElectronics).
El integrado ADS1298 es una buena opción como una mejora para una segunda iteración del diseño para mejorar la señal que se puede obtener debido a que este posee mayor tolerancia al ruido. Se debe destacar esta última opción debido a que se debe encargar directamente desde Texas Instruments y esto puede tomar mucho tiempo.

\newpage
\subsection{Temperatura}
En primera instancia se va a utilizar el sensor Lilypad ya que este está diseñado específicamente para wearables además de que utiliza un hilo conductor para unir sus terminales con la alimentación y la toma de datos. Este sensor tiene un valor aproximado de 3.790 CLP.\\
Dependiendo de los resultados obtenidos en la primeras pruebas se va a evaluar la segunda alternativa de utilizar el DS18B20 el cual tiene un valor aproximado de 5.900 CLP.
\subsection{IMU}
Al buscar las alternativas que existen en el país, todas las opciones de desarrollo usan distintas placas de desarrollo pero utilizan el mismo sensor MPU-9250 por lo que se va a utilizar la placa de Sparkfun MPU-9250 luego de tener el prototipo funcional con los primeros sensores de electrocardiograma y temperatura. Esta placa de desarrollo tiene un valor aproximado de 12.500 CLP.
\subsection{Comunicación}
Entre las alternativas mencionadas, se opta por utilizar en primera instancia el GPRS/GSM shield, el cual posee un costo aproximado de 43.980 CLP.
Cabe mencionar que en una segunda instancia se utilizaría el chip RN4020, el cual posee un precio aproximado de 20.000 CLP.\\
Si bien el costo es menor, es relevante la complejidad que agrega la segunda opción (al agregar un actor como lo puede ser un teléfono inteligente). Aunque la flexibilidad que brinda esta segunda opción y ciertos aspectos de la primera son las que obligan este cambio.

%Capítulo 5: Sistema de telecomunicaciones
\newpage
\chapter{Sistema de telecomunicaciones}\label{comunicacion}
Se determinó utilizar la plataforma Arduino por su simplicidad en prototipado y programación, además de ser de fácil acceso y una tecnología escalable. En base a esto se decide comenzar a trabajar en el apartado de comunicación.
\section{Redes móviles, Bluetooth y Android}
Como se comentó anteriormente, la primera elección de tecnología para la comunicación fue la de utilizar redes móviles directamente, esto por medio del módulo GPRS Shield. Este es un módulo de comunicación 2G compatible con socket XBee para la Arduino UNO (un socket XBee dota a una placa Arduino de la capacidad de comunicarse en forma inalámbrica \cite{xbee_info}). Para comodidad se escogió una variante de Arduino UNO llamado PICARO+, la cual posee entre otras modificaciones el socket XBee integrado.\\
Para esta comunicación se debe hacer uso tanto de la placa principal, el chip de comunicación y una antena, estas últimas 2 mencionadas en las figuras \ref{gprs} y \ref{antena}.\\
Dentro de las características del GPRS se encuentran el emplear una tarjeta SIM, conector SMA y el chip Quectel M95. En cuanto a la antena nos encontramos con cuatri-banda: 
\begin{enumerate}
	\item\textbf{GSM/850E: 824 a 894 [MHz]}
	\item\textbf{GSM: 880 a 960 [MHz]}
	\item\textbf{DCS: 1710 a 1880 [MHz]}
	\item\textbf{PCS: 1850 a 1990 [MHz]}
\end{enumerate}

\newpage
Si bien puede parecer cuestionable el utilizar tecnología 2G, es importante considerar que chip provee de hasta 85.6 [kbps] y diversos protocolos de comunicación. Con lo cual al año 2017 (se espera deshabilitar las redes 2G en el mediano plazo para dar paso a nuevas tecnologías) sirve como prueba de concepto dado su bajo costo y el acercamiento que ofrece a los comandos AT, los cuales son los empleados para controlar chips de este tipo.
Luego de comenzado el proceso de configuración, se encontraron diversos problemas con esta elección:

\begin{enumerate}
	\item\textbf{Dimensiones:}
	Dado que este módulo esta contemplado para operar en conjunto con la placa principal, se hace engorroso el tener una antena de casi 6 [cm] y de gran grosor adosado al cuerpo del paciente.
	\item\textbf{Consumo energético:}
	Este módulo hace necesario el uso de una fuente de alimentación externa de mayor capacidad (9[V] aproximadamente) respecto a la necesidad base de la placa (3.3[V]), lo que conlleva a usar un cargador externo y en su momento a una batería de mayor capacidad.
	\item\textbf{Antigüedad de comandos:}
	Los comandos Hayes (también llamados AT \cite{AT}) son un conjunto de comandos empleados en la configuración y parametrización de módems, su uso data de al menos 1990 y en cierto punto dejaron de usarse para dar paso a controladores específicos. 
	\item\textbf{Tasas de transferencias:}
	A raíz de un estudio preliminar en tasas de transferencia se estableció que alrededor de 150 datos por segundo debían ser enviados (esto en función de las tasas de operación de un ECG común\cite{ecg_rate}), por tanto el usar esta tecnología obliga a emplear 3G como mínimo. 
	\item\textbf{Costo:}
	En comparación a otras tecnologías indirectas de redes móviles como lo puede ser el Bluetooth, la inversión necesaria es mayor y su flexibilidad bastante menor.
\end{enumerate}

Por todo lo anterior, se pasa a una segunda iteración en busca de emplear tecnología Bluetooth y un intermediario para llegar a las redes celulares.\\


\section{Perfiles Bluetooth}

Bluetooth \cite{bluetooth} es una especificación industrial para Redes Inalámbricas de Área Personal (WPAN) creado por Bluetooth Special Interest Group, Inc. que posibilita la transmisión de voz y datos entre diferentes dispositivos mediante un enlace por radiofrecuencia en la banda ISM de los 2.4 GHz y data de 1994 y actualmente se encuentra en su versión 5.0. La versión a emplear en este proyecto es la 4.0 llamada BLE (Bluetooth Low Energy) que se detalla en capítulos adelante. \\

Un perfil Bluetooth es la especificación de una interfaz de alto nivel para su uso entre dispositivos Bluetooth. Para utilizar Bluetooth, un dispositivo debe implementar alguno de los perfiles soportados.
Los perfiles son descripciones de comportamientos generales que los dispositivos pueden utilizar para comunicarse, formalizados para favorecer un uso unificado. La forma de utilizar las capacidades de Bluetooth se basa, por tanto, en los perfiles que soporta cada dispositivo. Los perfiles permiten la manufactura de dispositivos que se adapten a sus necesidades.\\

A la fecha existen más de 27 perfiles Bluetooth, pero durante el desarrollo del proyecto se emplearon solo dos: SPP y GATT, los cuales se detallarán en su momento.

Las principales diferencias entre estos últimos dos perfiles son: GATT pertenece al estándar introducido en la versión 4.0 (desde ahora BLE) mientras que SPP en la versión 2.1. BLE está pensado para operar con un consumo energético inferior que versiones anteriores, posee mayor velocidad en el establecimiento de la conexión y está pensado para la transferencia de pequeñas cantidades de datos. Excepto por el último punto se puede observar una notoria superioridad de GATT (BLE) frente a SPP, pero como se verá más adelante, las tasas de transferencias obtenidas con GATT son lo suficientemente buenas como para escogerla en este proyecto.

\newpage
\section{Razones para Android}

Para seleccionar el intermediario entre la comunicación Bluetooth y las redes móviles se decidió un teléfono inteligente, por sus capacidades de cómputo, gran accesibilidad y flexibilidad al ofrecer un entorno de desarrollo propio de su sistema operativo. Ahora bien, para seleccionar el sistema operativo se recurre a su penetración en el mercado y como se puede observar en la figura \ref{market_share} Android se alza como el gigante en el mercado.

\begin{figure}[H]
	\centering
	\includegraphics[scale=0.4]{figuras/comunicacion/Market_share.png}
	\caption{Mercado compartido mundial de sistemas operativos móviles 2017-2018 \cite{market_share_cita}}
	\label{market_share}
\end{figure}

Además de lo anterior, se ha de considerar el entorno de desarrollo y el ecosistema que rodea al sistema operativo en cuestión. En el caso de Android, se trabaja principalmente con Android Studio (Para sistemas Linux, Windows y Mac), en lenguaje Java o Kotlin, con una comunidad activa de desarrolladores y una gran cantidad de librerías a disposición.\\
Por último se considera la accesibilidad de los terminales, con lo cual es de conocimiento general que Apple posee precios más elevados que los dispositivos Android.
Por lo tanto se concluye que Android es la mejor alternativa en estos momentos.

\newpage
\section{Comparativa desarrollo híbrido}
Para el desarrollo móvil actual existen dos grandes aproximaciones, las cuales pasan principalmente por el uso de entornos de desarrollo que permiten el despliegue en más de un sistema operativo (llamados híbridos) o uno determinado con un solo código fuente.\\

Primero, en el ámbito nativo (un código fuente para un despliegue único): 

\begin{table}[H]
	\centering
	\begin{tabular}{| c | c | c | c |}
		\hline
		\multicolumn{1}{|c|}{\textbf{OS}}&
		\multicolumn{1}{c|}{\textbf{Lenguaje}}&
		\multicolumn{1}{|c|}{\textbf{IDE}}&
		\multicolumn{1}{|c|}{\textbf{Plataforma}}\\ \hline
		Android  & Java, Kotlin  & Android Studio, Eclipse & Linux, Mac, Windows \\ \hline
		iOS  & Objective-C, Swift & XCode & Mac \\ \hline
	\end{tabular}
	\caption{Comparativa de desarrollo nativo, elaboración propia}
	\label{native_comparative}
\end{table}

Continuando con el ámbito híbrido, se destacan 3 grandes competidores:

\begin{table}[H]
	\centering
	\begin{tabular}{| c | c | c |}
		\hline
		\multicolumn{1}{|c|}{\textbf{Framework}}&
		\multicolumn{1}{c|}{\textbf{Tipo de resultado}}&
		\multicolumn{1}{|c|}{\textbf{Lenguaje}}\\ \hline
		Ionic  & No nativo  & JavaScript (AngularJS) \\ \hline
		Reac Native  & Nativo & JavaScript (React) \\ \hline
		Flutter  & Nativo & Dart \\ \hline
	\end{tabular}
	\caption{Comparativa de desarrollo híbrido, elaboración propia}
	\label{hybrid_comparative}
\end{table}

Por último se analizan ventajas y desventajas de ambas aproximaciones al desarrollo móvil, cabe destacar que Flutter aún en 2018 se encuentra en fase beta, pero se considera por las grandes prestaciones que presenta (por lo que no se considerarán sus ventajas en la siguiente tabla), además se establece un marco en donde se espera obtener un desarrollo tanto para iOS como para Android:

\begin{table}[H]
	\centering
	\begin{tabular}{| c | c | c |}
		\hline
		\multicolumn{1}{|c|}{\textbf{Característica}}&
		\multicolumn{1}{c|}{\textbf{Nativo}}&
		\multicolumn{1}{|c|}{\textbf{Híbrido}}\\ \hline
		Rendimiento  & Máximo  & Suficiente \\ \hline
		Actualizaciones SO  & Sin retraso & Con retraso \\ \hline
		Librerías  & Extenso & Acotado \\ \hline
		Control  & Total & Parcial \\ \hline
		Tiempo de desarrollo  & Alto & Medio/bajo \\ \hline
		Cantidad de código  & Alto & Mínimo \\ \hline
		Diversidad de código  & Total & Unificado \\ \hline
		Complejidad  & Alta & Baja \\ \hline
	\end{tabular}
	\caption{Desarrollo nativo versus híbrido, elaboración propia}
	\label{native_hybrid}
\end{table}

Como se puede observar en la tabla \ref{native_hybrid}, el mayor potencial para el desarrollo híbrido es cuando no se tienen funcionalidades demasiado específicas (que requieran librerías especiales), no se requiere gran rendimiento, no se utilizarán las últimas características de seguridad del SO y el tiempo es primordial.\\
Si bien se podría considerar el uso híbrido, se espera que la aplicación haga uso de alto poder de procesamiento, utilice librerías específicas (como graficar en tiempo real como se verá más adelante) y se tenga el mayor control posible de todos los procesos. Por lo tanto se descarta el uso de entornos híbridos para el desarrollo, en desmedro de la compatibilidad con dispositivos Apple.


\newpage
\section{Prueba de concepto}

Para comenzar el desarrollo e iniciar los sucesivos Sprint (metodología SCRUM), se hizo uso del perfil SPP de Bluetooth, incluído en su versión 2.1 + EDR (2004) y que permite comunicación bidireccional. Es uno de los perfiles fundamentales de Bluetooth al tener un comportamiento muy parecido a los de la comunicación serial (como la usada en conexiones RS-232 o UART).
Está basado en el protocolo RFCOMM y emula una linea serial, para su uso se utilizan dos actores, uno que actúa como servidor y otro que actúa como cliente. El primero queda a la espera de alguna conexión entrante (visible), luego por medio de una búsqueda y el uso de un UUID el segundo genera una conexión para comenzar a intercambiar datos.

El objetivo es generar una prueba de concepto por el cual se usara a una aplicación Android como puente para llevar información internet.

Para esto se implementó la siguiente arquitectura:

\begin{figure}[H]
	\centering
	\includegraphics[scale=0.4]{figuras/comunicacion/prueba.png}
	\caption{Arquitectura de la prueba de concepto. Fuente: Elaboración propia}
	\label{prueba_concept}
\end{figure}

Como se puede observar en la figura \ref{prueba_concept} el servidor Bluetooth ()desarrollado en Java) fue implementado en computador (Windows), mientras que la aplicación Android básica permite el escaneo, selección y conexión con el servidor Bluetooth. Esto último sin utilizar librerías externas.

La prueba resultó exitosa, pudiendo enviar información (cadenas de texto) desde el computador hasta una página web previamente configurada, la cual se detallará en el siguiente capítulo. Cabe destacar que el uso de este perfil Bluetooth fue solo por simplicidad y próximamente se hará uso de un perfil acorde al proyecto.

%Capítulo 6: Implementación de la solución de lado del servidor
\newpage
\chapter{Implementación de la solución de lado del servidor}\label{servidor}

%Capítulo 7: Implementación de la solución de lado del cliente Android
\newpage
\chapter{Implementación de la solución de lado del cliente Android}\label{servicios}

\section{Servicios en Android}

La necesidad de utilizar servicios para el proyecto es que no necesariamente se requerirá de interacción con el usuario para ciertas tareas, como el acoplamiento automático del dispositivo y el celular, la búsqueda de servicios Bluetooth o la comunicación del servidor hacia y desde el dispositivo. Es por esta razón que un subproceso no cumple con lo requerido, ya que solo podría existir dentro del tiempo en que algún usuario haga uso de la aplicación (la mantenga abierta).

Los Servicios\cite{services} en Android se pueden comportar de tres maneras a grandes rasgos:
	
	-Servicio iniciado: Proceso en segundo plano con una tarea preestablecida y que al cabo de ejecutar se detendrá, normalmente no retorna algo y no interactúa con la actividad que lo generó.

	-Servicio de enlace: Proceso en segundo plano que se enlaza de componentes para existir, permite tareas que retornan valores y es capaz de interactuar con una o más componentes, incluso bajo comunicación entre procesos (IPC). Se destruye si la no hay componentes enlazados.

	-Servicio iniciado y de enlace: Servicio que puede ejecutarse de forma indefinida (tarea preestablecida) y que permite enlaces de componentes para interactuar y comunicarse. Es la unión de ambas características anteriores en un mismo servicio.


\newpage

\section{Implementación de servicios y su comunicación en Android en función de los requerimientos del proyecto}

Para el apartado asociado a los servicios en Android se comienzan las pruebas con el fin de conseguir las necesidades del proyecto: Autoconexión y envío de datos por internet. Ambos procesos deben ser llevados a cabo sin intervención del usuario y de forma permanente en cuanto la aplicación esté instalada y configurada a un equipo (placa principal con módulos de sensores). \newline
Para esto se trabaja con la idea de un servicio que contenga dos procesos (no necesariamente hilos, dado que hay que detallar el funcionamiento del dispositivo BLE para la comunicación de sus datos). Por esta razón se analiza el funcionamiento de las clases Services e Intent Services, en donde su diferencia radica en el objetivo de su ejecución: Services para ejecución indefinida hasta su detención manual e Intent Services para ejecución de una tarea concreta (contempla la creación propia de un hilo de ejecución definida). Luego para el tema de comunicación con el servicio existen distintas alternativas: AIDL, Binder y Messenger. AIDL usado para comunicación entre procesos de forma primitiva, Binder para comunicación solo con la aplicación contenedora del servicio (ventaja de poder acceder directamente a métodos públicos del servicio) y Messenger que es una forma de comunicación entre procesos con estructura basada en AIDL pero de mayor nivel y facilidad de uso por medio de un mensajero. Esta tarea queda en espera para trabajar con el dispositivo BLE, configurando sus parámetros e integrando su funcionamiento al módulo central de procesamiento Arduino, aunque se deja esbozado el servicio a necesitar, mezclando un servicio de la clase Services creado solo una vez para luego que las próximas ejecuciones de la aplicación se enlacen a este servicio (definiendo así un servicio indefinido contenedor y manipulador de la interacción con el bluetooth a través de IBinder).



%Capítulo 8: Configuración RN4020
\newpage
\chapter{Configuración RN4020}\label{rn4020}
%\newpage



%Capítulo 9: Integración de las componentes de la solución
\newpage
% ---------------------------------------------------------------------------------------
\chapter{Integración de las componentes de la solución}\label{mldp}





%Capítulo 10: Prototipo funcional V1
\newpage
% ---------------------------------------------------------------------------------------
\chapter{Prototipo funcional V1}\label{proto1}

%Capítulo 11: Prototipo funcional V2
\newpage
\chapter{Prototipo funcional V2}\label{proto2}

Luego de un primer prototipo, se analiza el funcionamiento general para planificar cambios y mejoras necesarias. Entre las que destacan un control sobre el envío de datos desde los sensores y una optimización de código en Arduino (relevante por la baja frecuencia de operación que posee frente a las exigencias del proyecto).

Los detalles se presentan a continuación, aunque adelantando que los cambios principales de esta nueva versión fueron realizados a la programación del microcontrolador y su sincronización con la aplicación.

\section{Control de sensores por Android}

Un aspecto fundamental para el proyecto es la posibilidad de controlar el inicio y el fin de las mediciones desde la aplicación, permitiendo en una iteración posterior su control desde la web.

Para este efecto, se desarrolla un protocolo de comunicación con el cual se espera sincronizar las acciones entre los dos principales entes (microcontrolador con módulo Bluetooth y aplicación Android).

Se comienza estableciendo un conjunto de instrucciones homogéneas con las cuales los dispositivos puedan ejecutar órdenes:

\textbf{00}: Detener medición

\textbf{10}: Iniciar solo medición de ECG

\textbf{01}: Iniciar solo medición de T

\textbf{11}: Iniciar ambas mediciones \newpage

Como se puede observar, se utiliza un sistema binario en el cual con 2 bit se logran controlar 4 estados fundamentales. Se escoge esta arquitectura dada su fácil escalabilidad y sencilla comprensión al asignar 0 o 1 a cierta posición asociada (y preacordada) a algún sensor.

Esto en conjunto con un Handler (ejecución de una tarea en otro hilo) en Android, permite el control temporal de estas instrucciones. Logrando así que al momento que la aplicación recibe una petición de medicón, ésta sea comandada por el Handler que a su vez tiene un Timer asociado, con el cual se podrá controlar la ejecución de la medición por una cantidad de tiempo preestablecida.

\begin{figure}[H]
	\centering
	\includegraphics[scale=0.5]{figuras/proto2/handler.png}
	\caption{Control de peticiones por Handler y repetición}
	\label{handler}
\end{figure}

En la figura \ref{handler} se pueden observar los distintos estados asociados a una medición: Petición, medición, detención, conteo, verificación y término.

Al acabar con la medición, el microcontrolador comunica la cantidad de paquetes enviados asociados a cada sensor, cantidad que es verificada con los recibidos en la aplicación y dependiendo de esto último se repite la petición de forma automática o se da por terminada la medición. \newpage

Las secuencias de control corresponden a una cadena de caracteres compuesto por AA y FF al inicio y término respectivamente de una medición, seguidos por 4 caracteres contenedores de 2 byte hexadecimales con el número de paquetes enviados (0 a 65536 paquetes).

Esto último supone una restricción en la cantidad máxima de paquetes enviados por medición:

\textbf{ECG}: Con una tasa de 150 [Hz] | Máximo 6.8 minutos por medición.

\textbf{T}: Con una tasa de 1 [Hz] | Máximo 18.2 horas por medición.

\section{Renovación de servidor Arduino}

Considerando la opción de microcontrolador escogido (Arduino UNO), se tiene una limitante importante en cuanto a la potencia de procesamiento presente, la cual se limita a la frecuencia de operación con la que cuenta el chip ATmega328 que es de 16 [MHz]. 

Por esta razón se hace indispensable una programación prolija y con miras en la optimización en la ejecución de instrucciones. En este sentido, se realizan diversos cambios al código fuente del microcontrolador (se pueden ver con detalle en los Anexos):

Control del ciclo principal de operación bajo milisegundos a microsegundos.

Uso de punteros en vez de copia de memoria para el manejo de arreglos de char.

Uso de condiciones según precedencia y uso de else if según casos más frecuentes.

Control de estados por variables, minimizando verificaciones innecesarias.

\newpage

\section{Análisis de resolución y frecuencia para ECG}

Pasando directamente al manejo del ECG por parte del controlador, se realizan pruebas para establecer frecuencias necesarias y verificar nivel de resolución mínimas, entre las que se destacan los siguientes resultados.

\begin{figure}[H]
	\centering
	\includegraphics[scale=0.4]{figuras/proto2/8bit.png}
	\caption{ECG a 300 [Hz] y 8 bit de resolución}
	\label{8bit}
\end{figure}

\begin{figure}[H]
	\centering
	\includegraphics[scale=0.4]{figuras/proto2/10bit.png}
	\caption{ECG a 300 [Hz] y 10 bit de resolución}
	\label{10bit}
\end{figure}

\newpage

Como se puede obervar en las figuras anteriores, la resolución obtenida utilizando 8 bit o 10 bit no representa un cambio significativo a nivel visual, mientras que a nivel de procesamiento y almacenamiento sí posee gran injerencia (valor máximo: 256 y 1024 respectivamente). 

Esto es especialmente importante cuando se tiene en cuenta que la comunicación hexadecimal funciona de forma óptima con información condensada en múltiplos de 8 bit, puesto que la representación de 1 byte (8 bit) es posible por medio de solo 2 caracteres hexadecimales que son exactamente el byte de información.

Así, se decide emplear esta resolución para la captura y comunicación de los datos de ECG, sin miedo a perder información o tener sobrecarga de caracteres, en una comunicación con paquetes contenedores más grandes que la información mínima.

Se hicieron pruebas de frecuencia para el ECG, probando valores entre 50 [Hz] y 1.000 [Hz]. Como resultado de esto se estableció un mínimo de 50 [Hz] de operación, una base esperable de 150 [Hz] y un máximo de 300 [Hz], los anteriores valores por condensación visual de los datos (se observa una compresión horizontal de las gráficas a menores frecuencias y lo contario para frecuencias mayores).

\begin{figure}[H]
	\centering
	\includegraphics[scale=0.4]{figuras/proto2/1000hz.png}
	\caption{ECG a 1.000 [Hz] y 10 bit de resolución}
	\label{1000hz}
\end{figure}

\section{Base de datos local (comparativa Realm, SQLite) }

De nada sirve obtener y comunicar datos si estos no son almacenados y estudiados posteriormente, es por ello que se contempló desde el inicio del proyecto el uso de una base de datos local para la aplicación (útil especialmente para mediciones sin cobertura) y una base de datos propia del servidor web.

En Android existen principalmente dos grandes alternativas respecto a bases de datos: SQLite y Realm (ambas relacionales y con esquema llave-valor). La primera de ellas, SQLite es un motor de base datos ampliamente utilizada y disponible desde los inicios del SO Android, sus grandes ventajas son la gran penetración que ya posee en los desarrolladores y los distintos sabores entenidos en las librerías que lo implementan.

La segunda alternativa es relativamente nueva pero con un potencial enorme, dada su facilidad de uso, gran rendimiento, visualización de datos con aplicaciones externas, disponibilidad en múltiples SO y su excelente documentación.

A continuación se presentan gráficas provistas por el equipo de Realm en comparativas de rendimiento \cite{realm_android} frente a distintas librerías en Android (en todas más alto es mejor).

 \begin{figure}[H]
 	\centering
 	\includegraphics[scale=0.3]{figuras/proto2/benchmark.png}
 	\caption{Benchmark frente a SQLite con el uso de distintas librerías}
 	\label{benchmark_realm}
 \end{figure}

\begin{figure}[H]
	\centering
	\includegraphics[scale=0.4]{figuras/proto2/insert.png}
	\caption{Comparativa en escrituras a la base de datos}
	\label{insert}
\end{figure}

\begin{figure}[H]
	\centering
	\includegraphics[scale=0.4]{figuras/proto2/queries.png}
	\caption{Comparativa en lecturas a la base de datos}
	\label{queries}
\end{figure}

Es por las figuras anteriores y su resultado apabullante que se hace uso de Realm y no de otra base de datos local. Cabe mencionar que al ser más reciente, posee características interesantes como un manejo sencillo de hilos, instancias, creación de arreglo de objetos, entre otros.

\newpage

\section{Modelo tentativo para Realm}

Si bien no se alcanza a implementar la base de datos ya escogida, se articula un modelo tentativo de las tablas que contendrán la información de las mediciones. Estableciendo de esta manera los datos y la relación presente entre ellos, se presenta la llave primaria en negrita para cada tabla.

\begin{table}[H]
	\centering
	\begin{tabular}{| c | c |}
		\hline
		\multicolumn{1}{|c|}{\textit{Dato}}&
		\multicolumn{1}{c|}{\textit{Tipo}}\\ \hline
		\textbf{Rut}  & String   \\ \hline
		Nombre  & String  \\ \hline
		Sexo & String  \\ \hline
		MAC & String  \\ \hline
		Hospital & String  \\ \hline
		Código Sistema & int  \\ \hline
		Edad & int  \\ \hline
		Teléfono & int  \\ \hline
		Teléfono emergencia & int  \\ \hline
		Teléfono emergencia 2 & int  \\ \hline
		Nº Ficha & int  \\ \hline
	\end{tabular}
	\caption{Tabla Paciente: Datos personales del paciente}
	\label{tabla_paciente}
\end{table}

\begin{table}[H]
	\centering
	\begin{tabular}{| c | c |}
		\hline
		\multicolumn{1}{|c|}{\textit{Dato}}&
		\multicolumn{1}{c|}{\textit{Tipo}}\\ \hline
		\textbf{Rut}  & String   \\ \hline
		\textbf{Fecha\_hora\_min\_seg}  & String  \\ \hline
		Duración & int  \\ \hline
		Sensores & int  \\ \hline
	\end{tabular}
	\caption{Tabla Mediciones: Al iniciar una medición almacena los datos relacionada a esta por posible retransmisión necesaria hacia el servidro web}
	\label{tabla_mediciones}
\end{table}

\begin{table}[H]
	\centering
	\begin{tabular}{| c | c |}
		\hline
		\multicolumn{1}{|c|}{\textit{Dato}}&
		\multicolumn{1}{c|}{\textit{Tipo}}\\ \hline
		\textbf{Rut}  & String   \\ \hline
		\textbf{Fecha\_hora\_min\_seg}  & String  \\ \hline
		Duración & int  \\ \hline
		Enviado & boolean  \\ \hline
	\end{tabular}
	\caption{Tabla ECG: Al iniciar una medición de ECG}
	\label{tabla_ECG}
\end{table}

\begin{table}[H]
	\centering
	\begin{tabular}{| c | c |}
		\hline
		\multicolumn{1}{|c|}{\textit{Dato}}&
		\multicolumn{1}{c|}{\textit{Tipo}}\\ \hline
		\textbf{Fecha\_hora\_min\_seg}  & String  \\ \hline
		\textbf{Contador}  & int  \\ \hline
		Valor & int  \\ \hline
	\end{tabular}
	\caption{Tabla Datos ECG: Se escribe con cada dato, pero asociado a una medición(Fecha\_hora\_min\_seg)}
	\label{tabla_datos_ECG}
\end{table}

\begin{table}[H]
	\centering
	\begin{tabular}{| c | c |}
		\hline
		\multicolumn{1}{|c|}{\textit{Dato}}&
		\multicolumn{1}{c|}{\textit{Tipo}}\\ \hline
		\textbf{Rut}  & String   \\ \hline
		\textbf{Fecha\_hora\_min\_seg}  & String  \\ \hline
		Duración & int  \\ \hline
		Enviado & boolean  \\ \hline
	\end{tabular}
	\caption{Tabla T: Al iniciar una medición de T}
	\label{tabla_T}
\end{table}

\begin{table}[H]
	\centering
	\begin{tabular}{| c | c |}
		\hline
		\multicolumn{1}{|c|}{\textit{Dato}}&
		\multicolumn{1}{c|}{\textit{Tipo}}\\ \hline
		\textbf{Fecha\_hora\_min\_seg}  & String  \\ \hline
		\textbf{Contador}  & int  \\ \hline
		Valor & int  \\ \hline
	\end{tabular}
	\caption{Tabla Datos T: Se escribe con cada dato, pero asociado a una medición(Fecha\_hora\_min\_seg)}
	\label{tabla_datos_T}
\end{table}



%Capítulo 12: Prototipo final
\newpage
\chapter{Prototipo final}\label{protof}


%Capítulo 13: Discusión
\newpage
\chapter{Discusión}\label{discusion}

%Capítulo 14: Conclusiones
\newpage
\chapter{Conclusiones}\label{conclusion}

Al término de este documento, se da por finalizado el trabajo asociado al programa de memorias multidisciplinarias, habiendo logrado los resultados propuestos como equipo y obteniendo en el proces o grandes experiencias, tanto personales como profesionales.

El desarrollo en equipo del proyecto puso sobre la palestra distintos obstáculos, como la coordinación de tareas, organización de reuniones, distintos roles asociados a las distintas etapas y a la pérdida de un integrante en los inicios. Pese a esto, fue una grata oportunidad el poder compartir con distintas áreas de la ingeniería, con sus puntos de vista y formas de enfrentar desafíos. En todos estos aspectos ayudó en gran medida la posibilidad de utilizar metodologías ágiles como lo es SCRUM, por adecuarse de gran manera a cambios repentinos y entregas iterativas de resultados en función de requerimientos a su vez bastante volátiles.

Una de las motivaciones presentes en este proyecto era el uso de nuevas tecnologías aplicadas en un problema de innovación, en donde poder aplicar herramientas presentes y adaptarlas a las necesidades actuales de la industria. En conjunto con esto, el hecho de poder acercarse al área de la medicina y comprender los desafíos técnicos inherentes a trabajar con pacientes, es de las cosas más enriquecedoras que se pueden rescatar.

En el ámbito técnico, se observaron las distintas complicaciones que tienen proyectos de esta índole (innovación médica), como lo es la cantidad de derivaciones necesarias para ser aprovechadas por un profesional, la capacidad de cómputo necesaria y la gran cantidad de datos concernientes a solo una medición (además de las variables que afectan a estos datos y que son tan importantes como ellos).

\newpage

El uso de tecnologías emergentes y de gran impacto fue una de las aristas visibles desde el inicio del proyecto, pero de poco sirven si no están pensadas para ser utilizadas por el común de la población. Es en este sentido, que el costo de la solución era un tema de gran relevancia a lo largo de todo el proyecto, puesto que el foco siempre se centró en ofrecer una solución a hospitales públicos en lo posible. Si bien al término del proyecto no se logra obtener un producto como tal, nunca se dejó de lado este aspecto y se fue consecuente en las distintas decisiones tomadas a lo largo del desarrollo del proyecto.

El proyecto posee un costo fijo de alrededor de 50 USD solo por parte del hardware, otros 10 USD mensuales por obtener un servidor en Chile como el ya presentado y 15 USD anuales por el dominio del sitio web, haciendo un total de 135 USD anuales como costos anuales. Con lo anterior, se visualizan a grandes rasgos los costos logrados (sin considerar horas asociadas a los profesionales a cargo o mantenimiento de los mismos) y sumado a la utilización de tecnologías como lo son las redes celulares se dan por logrados los principales objetivos del proyecto: Bajo costo y alta cobertura bajo las limitantes geográficas del país.

Entre las proyecciones que se esperaban en un inicio estaba la posibilidad de generar un emprendimiento a partir del proyecto, aspecto que se intentó con la incubadora de la Universidad pero que no pudo continuar por las razones ya expuestas en su momento. 

%Capítulo 15: Anexos
\newpage
\input{./Capitulos/15_anexos}

%Referencias
\newpage
\renewcommand{\refname}{Referencias}
\begin{thebibliography}{99}

\bibitem{visi} Sotera Wireless, ViSi Mobile\textregistered  System, rev. 05 marzo 2018, \hyperref[visi]{http://www.soterawireless.com/visi-mobile/}

\bibitem{visi_tel} Sotera Wireless, ViSi Mobile\textregistered  System, rev. 15 Mayo 2018, \hyperref[visi_tel]{https://newatlas.com/visi-mobile-wireless-health-monitoring/25583/}

\bibitem{qardio} Qardio Inc., QardioCore, rev.  05 marzo 2018, \hyperref[qardio]{https://www.getqardio.com/es/qardiocore-wearable-ecg-ekg-monitor-iphone/}

\bibitem{qardio_tel} Qardio Inc., QardioCore, rev.  15 Mayo 2018,
\hyperref[qardio_tel]{https://store.getqardio.com/products/qardiocore}

\bibitem{nuubo} Nuubo, Nuubo wearable ECG, rev. 05 marzo 2018, \hyperref[nuubo]{https://www.nuubo.com/producto}

\bibitem{nuubo_tel} Nuubo, Nuubo wearable ECG, rev. 15 Mayo 2018, \hyperref[nuubo_tel]{http://pdf.medicalexpo.com/pdf/nuubo/necg-minder/83949-96239.html}

\bibitem{ad8232} Analog Devices, ``Single-Lead Heart Rate Monitor Front End'', Rev. B Marzo 2017, \hyperref[ad8232]{http://www.analog.com/media/en/technical-documentation/data-sheets/AD8232.pdf}

\bibitem{RN4020} Microchip, ``Bluetooth Low Energy Module RN4020'', Rev. 15 Mayo 2018, \hyperref[RN4020]{http://ww1.microchip.com/downloads/en/DeviceDoc/50002279B.pdf}

\bibitem{RF} Wikipedia, ``Radio Frecuencia'', Rev. 15 Mayo 2018, \hyperref[RF]{https://es.wikipedia.org/wiki/Radiofrecuencia}

\bibitem{satelite} Wikipedia, ``Comunicaciones por satélite'', Rev. 15 Mayo 2018, \hyperref[satelite]{https://goo.gl/ykf7tv}

\bibitem{wifi} Wikipedia, ``IEEE 802.11'', Rev. 15 Mayo 2018, \hyperref[wifi]{https://goo.gl/1hmiWW}

\bibitem{celular} Wikipedia, ``Red de celdas'', Rev. 15 Mayo 2018, \hyperref[celular]{https://goo.gl/K2sKoQs}

\bibitem{temp} Maxim Integrated, ``Programable Resolution 1-Wire Digital Thermometer'', rev Enero 2015, \hyperref[temp]{https://datasheets.maximintegrated.com/en/ds/DS18B20.pdf}

\bibitem{xbee_info} XBee, ``¿Qué es XBee?'', rev Mayo 2018, \hyperref[xbee_info]{http://xbee.cl/que-es-xbee/}

\bibitem{AT} Wikipedia, ``Conjunto de comandos Hayes'', rev Mayo 2018, \hyperref[AT]{https://goo.gl/xpCfKC}

\bibitem{ecg_rate} Fire EMS, ``Understanding ECG Filtering'', rev Mayo 2018, \hyperref[ecg_rate]{http://www.ems12lead.com/2014/03/10/understanding-ecg-filtering/}

\bibitem{bluetooth} Wikipedia, ``Bluetooth'', rev Mayo 2018, \hyperref[bluetooth]{https://es.wikipedia.org/wiki/Bluetooth}

\bibitem{market_share_cita} StatCounter, ``GlobalStats'', rev Mayo 2018, \hyperref[market_share_cita]{https://goo.gl/62GngF}

\bibitem{hosting_web} Wikipedia, ``Alojamiento Web'', rev Mayo 2018, \hyperref[hosting_web]{https://es.wikipedia.org/wiki/Alojamiento\_web}

\bibitem{services} Google, ``Servicios'', rev Julio 2018, \hyperref[services]{https://developer.android.com/guide/components/services?hl=es-419}

\bibitem{RN4020_code} Microchip, ``RN4020'', rev Julio 2018, \hyperref[RN4020_code]{https://www.microchip.com/wwwproducts/en/RN4020}

\bibitem{realm_android} Realm, ``Realm Android'', rev Julio 2018, \hyperref[realm_android]{https://realm.io/blog/realm-for-android}

\bibitem{user_guide_rn4020} Microchip, ``RN4020 User Guide'', rev Julio 2018, \hyperref[user_guide_rn4020]{http://ww1.microchip.com/downloads/en/DeviceDoc/70005191B.pdf}

\bibitem{java8} Android, ``Usar funciones del lenguaje de Java 8'', rev Julio 2018, \hyperref[java8]{https://developer.android.com/guide/platform/j8-jack}

\bibitem{proguard} ProGuard, ``The open source optimizer for Java bytecode'', rev Julio 2018, \hyperref[prroguard]{https://www.guardsquare.com/en/products/proguard}

\bibitem{flutter} Flutter, ``Build beautiful native apps in record time'', rev Julio 2018, \hyperref[flutter]{https://flutter.io}

\bibitem{kotlin} Kotlin, ``Statically typed programming language for modern multiplatform applications'', rev Julio 2018, \hyperref[kotlin]{https://kotlinlang.org}


\bibitem{http2sse} InfoQ, ``Will WebSocket survive HTTP/2?'', rev Julio 2018, \hyperref[http2sse]{https://www.infoq.com/articles/websocket-and-http2-coexist}

\bibitem{unittest} Android Developers, ``Build effective unit tests'', rev Julio 2018, \hyperref[unittest]{https://developer.android.com/training/testing/unit-testing}

\bibitem{sonarqube} SonarQube, ``Continuous
Code Quality'', rev Julio 2018, \hyperref[sonarqube]{https://www.sonarqube.org}

\bibitem{tensor} TensorFlow, ``An open source machine learning framework for everyone'', rev Julio 2018, \hyperref[tensor]{https://www.tensorflow.org}


\bibitem{mxnet} SonarQube, ``A flexible and efficient library for deep learning'', rev Julio 2018, \hyperref[mxnet]{https://mxnet.apache.org}

\bibitem{jenkins} Jenkins, ``Build great things at any scale'', rev Julio 2018, \hyperref[jenkins]{https://jenkins.io}

\bibitem{nodejs} NodeJS, ``NodeJS'', rev Julio 2018, \hyperref[nodejs]{https://nodejs.org/es}

\bibitem{socket} Socket.io, ``FEATURING THE FASTEST AND MOST RELIABLE REAL-TIME ENGINE'', rev Julio 2018, \hyperref[socket]{https://socket.io/}

\bibitem{angularjs} AngularJS, ``AngularJS'', rev Julio 2018, \hyperref[angularjs]{https://angularjs.org/}

\bibitem{expressjs} ExpressJS, ``Infraestructura web rápida, minimalista y flexible para Node.js'', rev Julio 2018, \hyperref[expressjs]{http://expressjs.com/es/}

\bibitem{meteor} MeteorJS, ``THE FASTEST WAY TO BUILD JAVASCRIPT APPS'', rev Julio 2018, \hyperref[meteor]{https://www.meteor.com/}

\bibitem{mongoose} Mongoose, ``Elegant mongodb object modeling for node.js'', rev Julio 2018, \hyperref[mongoose]{http://mongoosejs.com/}

\bibitem{influxdb} InfluxDB, ``The modern engine for Metrics and Events'', rev Julio 2018, \hyperref[influxdb]{https://www.influxdata.com/}

\bibitem{firebase} FireBase, ``Firebase te ayuda a crear mejores apps para dispositivos móviles y hacer crecer tu empresa.'', rev Julio 2018, \hyperref[firebase]{https://firebase.google.com/?hl=es-419}

\bibitem{postgresql} PostgreSQL, ``THE WORLD'S MOST ADVANCED OPEN SOURCE RELATIONAL DATABASE'', rev Julio 2018, \hyperref[prostgresql]{https://www.postgresql.org/}

\bibitem{grafana} Grafana, ``The open platform for beautiful analytics and monitoring'', rev Julio 2018, \hyperref[grafana]{https://grafana.com/}

\bibitem{d3} D3, ``D3: Data-Driven Documents'', rev Julio 2018, \hyperref[d3]{https://d3js.org/}

\bibitem{highcharts} HighCharts, ``Make your data come alive'', rev Julio 2018, \hyperref[highcharts]{https://www.highcharts.com/}

\bibitem{chartjs} ChartJS, ``Simple yet flexible JavaScript charting for designers \& developers'', rev Julio 2018, \hyperref[chartjs]{https://www.chartjs.org/}

\bibitem{websocket} Wikipedia, ``WebSocket'', rev Julio 2018, \hyperref[websocket]{https://es.wikipedia.org/wiki/WebSocket}

\bibitem{namecheap} Namecheap, ``Search for your domain name'', rev Julio 2018, \hyperref[namecheap]{https://www.namecheap.com/}

\bibitem{atmosphere} Atmosphere, ``Explore Meteor Packages'', rev Julio 2018, \hyperref[atmosphere]{https://atmospherejs.com/}

\bibitem{ddp} Meteor, ``Server Connections'', rev Julio 2018, \hyperref[ddp]{https://docs.meteor.com/api/connections.html}

\bibitem{mongodb} MongoDB, ``MongoDB for GIANT Ideas'', rev Julio 2018, \hyperref[mongodb]{https://www.mongodb.com/}

\bibitem{nginx} NGINX, ``NGINX | High Performance Load Balancer, Web Server \& Reverse Proxy'', rev Julio 2018, \hyperref[nginx]{https://www.nginx.com/}

\bibitem{certbot} CertBot, ``Automatically enable HTTPS on your website with EFF's Certbot, deploying Let's Encrypt certificates.'', rev Julio 2018, \hyperref[certbot]{https://certbot.eff.org/}

\bibitem{nginx_certbot} SeaFile-docs, ``Enabling Https with Nginx'', rev Julio 2018, \hyperref[nginx_certbot]{https://manual.seafile.com/deploy/https\_with\_nginx.html}

\bibitem{mvvm} Erik Jhordan Rey, ``ESCRIBIENDO ANDROID APPS CON DATA BINDING'', rev Julio 2018, \hyperref[mvvm]{https://erikcaffrey.github.io/ANDROID-databinding-android/}

\bibitem{mvp} AntonioLeiva, ``MVP for Android: how to organize the presentation layer'', rev Julio 2018, \hyperref[mvp]{https://antonioleiva.com/mvp-android/}

\comm{



\bibitem{isp} Microchip, ``In-Circuit Serial Programming (ICSP) Guide'', rev Enero 2015, \hyperref[temp]{http://ww1.microchip.com/downloads/en/devicedoc/30277d.pdf}

\bibitem{ft232} FTDI Chip, ``Future Technology Devices International Ltd. FT232R USB UART IC'', rev 18 noviembre 2015, \hyperref[temp]{http://www.ftdichip.com/Support/Documents/DataSheets/ICs/DS\_ FT232R.pdf}

\bibitem{bateria} Microchip, ``Stand\- Alone System Load Sharing and Li\- Ion Baterry Charge Management Controller'', rev Septiembre 2013, \hyperref[temp]{http://ww1.microchip.com/downloads/en/DeviceDoc/20002090D.pdf}
}
%
%\bibitem{NETR/NETW} How is data exchanged between two SIMATIC S7-200 %devices in PPI mode?\\
%\hyperref[NETRNETW]{https://support.industry.siemens.com/cs/document/%370000/how-is-data-exchanged-between-two-simatic-s7-200-devices-in-ppi-%mode-?dti=0\&lc=en-WW} 
%
\end{thebibliography}
		
\end{document}