% ---------------------------------------------------------------------------------------
\chapter{Conclusiones}\label{chap9}
Al ver el esquem�tico terminado y las piezas seleccionadas se puede notar que se cumplen los requisitos propuestos en un principio. Realizando un listado de costos de componentes se obtiene un valor de $42 USD$ total. Adem�s el costo de fabricaci�n de una PCB de tama�o menor a $10x10[cms]$ es de aproximadamente $7 USD$. Lo que lleva a un costo total del dispositivo de $49 USD$ contando solamente componentes electr�nicas.\\
Es importante destacar que esta primera versi�n del dispositivo es solo un comienzo en la larga etapa de desarrollo, luego de producir esta primera placa se debe realizar las pruebas correspondientes y verificaci�n de integridad de la placa.\\
El dise�o final de la PCB se puede observar en los anexos.\\
Adem�s se aprendi� a dise�ar un dispositivo electr�nico desde su prototipo en una placa de desarrollo, donde fue necesario leer constantemente las hojas de datos de todas las componentes a utilizar y seleccionar cuales son las mejores, que al mismo tiempo sean compatibles y coherente con el dise�o que se est� llevando a cabo.\\
Tambi�n es importante destacar la experiencia que se obtuvo con el concurso de emprendimiento. Se aprendi� mucho del mundo del emprendimiento al presentar el trabajo realizado. Adem�s se tuvo la oportunidad de observar los trabajos propuestos por los otros emprendedores y estudiar los criterios de evaluaci�n. Se di� mucha importancia al modelo de negocios y las alianzas estrat�gicas que ten�a cada participante, mas que al trabajo realizado y estado actual de la idea o prototipo. Muchos de los seleccionados ya ten�an potenciales clientes y alianzas estrat�gicas que los hac�a mejores candidatos a ganar el dinero para el emprendimiento.