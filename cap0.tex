\section*{Agradecimientos}
Terminado este trabajo de t�tulo se cierra un largo ciclo Universitario, en el que he vivido distintas
experiencias que me han hecho crecer como profesional y como persona.

Quisiera agradecer a todos
los participantes de esta etapa, que son bastantes, demasiados para que pueda mencionarlos a todos,
sin embargo no quiero que olviden lo mucho que significaron para m� en este proceso. 

Quiero agradecer a mi padre y a mi madre, quienes han estado siempre para ayudarme, me han
apoyado en mis decisiones y han dado todo por m�. Su fe en m� y su apoyo continuo me dieron las
fuerzas para enfrentar los desaf�os que se me presentaban.
A mi hermana, mi familia, mis amigos, mis compa�eros, gracias por su preocupaci�n, su ayuda,
su apoyo y sus ense�anzas.

\newpage
\section*{Resumen}
Este trabajo consistir� en el desarrollo de un dispositivo electr�nico para monitoreo remoto de pacientes con distintos sensores que sea usable, de bajo costo y aut�nomo con env�o de informaci�n mediante conexi�n bluetooth de bajo consumo. 
Para lograr lo anterior se comenzar� buscando alternativas existentes en el mercado y se desarrollar� un prototipo funcional para evaluar a partir de las componentes disponibles. Finalmente a partir del prototipo funcional se realizar� el dise�o del esquem�tico y el circuito impreso en el programa EagleCAD para ser enviado a producci�n.\par
Es importante destacar que para el dise�o del dispositivo electr�nico se espera mantener el est�ndar de tecnolog�as que existen en la actualidad como el uso de bater�a interna recargable y que sea de f�cil uso para el usuario.
\newpage

%\section*{Abstract}

%\newpage

\section*{Glosario}

\begin{tabular}{lcp{10.5cm}}
Esquem�tico &:& Es una representaci�n pict�rica de un circuito electr�nico. Muestra las diferentes componentes del circuito de manera simple y las conexiones de alimentaci�n y se�ales entre distintos dispositivos.\\
PCB &:& Printed Circuit Board, El circuito impreso se utiliza para conectar el�ctricamente a trav�s de las pistas conductoras, y sostener mec�nicamente, por medio de la base, un conjunto de componentes electr�nicos.\\
Hardware &:& Partes f�sicas tangibles de un sistema inform�tico.\\
Software &:& Aplicaciones o programas que funcionan en un sistema inform�tico.\\
Firmware &:& Programa inform�tico que establece la l�gica de mas bajo nivel que controla los circuitos electr�nicos.\\
Bootloader &:& Es un programa que no tiene la totalidad de las funcionalidades para operar un sistema y est� dise�ado para preparar todo lo que necesita el firmware para ejecutarse.\\
IIH &:& Infecci�n Intra-Hospitalaria
\end{tabular}

%%%%%%%%%%%%%%%%%%
%\begin{tabular}{lcp{10.5cm}}
%Rel�&:& Interruptor controlado por un circuito el�ctrico en el que, por medio de una bobina y un electroim�n, se acciona un juego de uno o %varios contactos que permiten abrir o cerrar otros circuitos el�ctricos independientes.\\
%\end{tabular}
%%%%%%%%%%%%%%%%%%