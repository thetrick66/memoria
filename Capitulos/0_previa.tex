\section*{Agradecimientos}

Agradecer es un paso fundamental en todo desarrollo humano, puesto que es de las pocas oportunidades de reflexionar sobre quienes estuvieron y están a nuestro lado en alguna etapa de nuestra vida. Me gustaría destacar que aun cuando agradecer es una vista al pasado, no existe tiempo inconexo en el corazón y los llevo siempre conmigo.

Quiero agradecer a mi familia, mi pareja, amigos, compañeros, profesores y toda persona con quien he tenido contacto en esta etapa universitaria, todos me han formado y son parte de este trabajo de una o de otra manera.

Le dedico este trabajo a quienes siempre creyeron en mí y a quienes aun lo hacen. Porque incluso teniendo un núcleo familiar distinto, el amor y la comprensión siempre estuvieron conmigo: A mi padre Omar Bernales Vega, a mis madres Toya y Mónica Gatica y mis hermanas Bárbara, Elein y Maka. Son mi orgullo y mi ejemplo a seguir.

Por último y no por ello menos importante, a la persona que soportó mis rabietas y jornadas de estrés, quien aun me acompaña y ama de forma extraordinaria, mi Valeria.

\newpage

\section*{Resumen}

El presente documento relatará la resolución de un problema real y actual en Chile, a partir de un desafío propuesto en el contexto de las Memorias Multidisciplinarias.\par
El desafío consiste en el desarrollo de un sistema con la capacidad de monitorear pacientes de forma remota, de bajo costo y con las limitantes geográficas propias de nuestro país, teniendo en mente su aplicación a nivel público del Sistema de Salud. Para esto, se analizaron las distintas opciones existentes en el mercado y se desarrolló una solución a nivel de prototipo funcional que cumpliese con las restricciones ya mencionadas.\par
Por ser un desafío resuelto de forma multidisciplinaria es importante destacar que el desarrollo en este documento estará enfocado al área informática y de telecomunicaciones asociada a la adquisición, procesamiento, almacenamiento y envío de datos.\par
El resto del equipo final está compuesto por: Sebastián Castillo actual Ingeniero en Diseño de Productos y Felipe Cordero actual Ingeniero Civil Electrónico, ambos de la misma casa de estudios UTFSM. Ambas memorias complementan la actual en el ámbito correspondiente a sus carreras, pero lógicamente compartiendo su núcleo como proyecto conjunto.
\newpage

%\section*{Abstract}

%\newpage

\section*{Glosario}

\begin{tabular}{lcp{10.5cm}}
RF &:& Radio Frecuencia\\
	
Wi-Fi &:& Proviene del termino Wireless Fidelity. Corresponde a la norma IEEE 802.11
que define los estándares de conectividad inalámbrica para transmisión de datos entre dispositivos.\\


Esquemático &:& Es una representación pictórica de un circuito electrónico. Muestra las diferentes componentes del circuito de manera simple y las conexiones de alimentación y señales entre distintos dispositivos.\\
PCB &:& Printed Circuit Board, El circuito impreso se utiliza para conectar eléctricamente a través de las pistas conductoras, y sostener mecánicamente, por medio de la base, un conjunto de componentes electrónicos.\\
Hardware &:& Partes físicas tangibles de un sistema informático.\\
Software &:& Aplicaciones o programas que funcionan en un sistema informático.\\
Firmware &:& Programa informático que establece la lógica de mas bajo nivel que controla los circuitos electrónicos.\\
Bootloader &:& Es un programa que no tiene la totalidad de las funcionalidades para operar un sistema y está diseñado para preparar todo lo que necesita el firmware para ejecutarse.\\
IIH &:& Infección Intra-Hospitalaria
\end{tabular}

%%%%%%%%%%%%%%%%%%
%\begin{tabular}{lcp{10.5cm}}
%Relé&:& Interruptor controlado por un circuito eléctrico en el que, por medio de una bobina y un electroimán, se acciona un juego de uno o %varios contactos que permiten abrir o cerrar otros circuitos eléctricos independientes.\\
%\end{tabular}
%%%%%%%%%%%%%%%%%%