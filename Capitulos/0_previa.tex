\section*{Agradecimientos}

Agradecer es un paso fundamental en todo desarrollo humano, puesto que es de las pocas oportunidades de reflexionar sobre quienes estuvieron y están a nuestro lado en alguna etapa de nuestra vida. Me gustaría destacar que aún cuando agradecer es una vista al pasado, no existe tiempo inconexo en el corazón y los llevo siempre conmigo.

Quiero agradecer a mi familia, mi pareja, amigos, compañeros, profesores y toda persona con quien he tenido contacto en esta etapa universitaria, todos me han formado y son parte de este trabajo de una o de otra manera.

Le dedico este trabajo a quienes siempre creyeron en mí y a quienes aún lo hacen. Porque incluso teniendo un núcleo familiar distinto, el amor y la comprensión siempre estuvieron conmigo: A mi padre Omar Bernales Vega, a mis madres Toya y Mónica Gatica y mis hermanas Bárbara, Elein y Maka. Son mi orgullo y mi ejemplo a seguir.

Por último y no por ello menos importante, a la persona que soportó mis rabietas y jornadas de estrés, quien aún me acompaña y ama de forma extraordinaria, mi Valeria.

\newpage

\section*{Resumen}

El presente documento relatará la resolución de un problema real y actual en Chile, a partir de un desafío propuesto en el contexto de las Memorias Multidisciplinarias.\par
El desafío consiste en el desarrollo de un sistema con la capacidad de monitorear pacientes de forma remota, de bajo costo y con las limitantes geográficas propias de nuestro país, teniendo en mente su aplicación a nivel público del Sistema de Salud. Para esto, se analizaron las distintas opciones existentes en el mercado y se desarrolló una solución a nivel de prototipo funcional que cumpliese con las restricciones ya mencionadas.\par
Por ser un desafío resuelto de forma multidisciplinaria es importante destacar que el desarrollo en este documento estará enfocado al área informática y de telecomunicaciones asociada a la adquisición, procesamiento, almacenamiento y envío de datos.\par
El resto del equipo final está compuesto por: Sebastián Castillo actual Ingeniero en Diseño de Productos y Felipe Cordero actual Ingeniero Civil Electrónico, ambos de la misma casa de estudios UTFSM. Ambas memorias complementan la actual en el ámbito correspondiente a sus carreras, pero lógicamente compartiendo su núcleo como proyecto conjunto.
\newpage

%\section*{Abstract}

%\newpage

\section*{Glosario}

\begin{tabular}{lcp{10.5cm}}

2G &:& Segunda generación de telefonía móvil.\\	
	
RF &:& Radio Frecuencia.\\
	
Wi-Fi &:& Proviene del termino Wireless Fidelity. Corresponde a la norma IEEE 802.11
que define los estándares de conectividad inalámbrica para transmisión de datos entre dispositivos.\\

IDE &:& De sus siglas en inglés: Entorno de desarrollo integrado.\\

SO &:& Sistema Operativo.\\

Framework &:& Estructura conceptual y tecnológica de soporte definida, normalmente con artefactos o módulos de software concretos, en base a la cual otro proyecto de software puede ser organizado y desarrollado. Típicamente, puede incluir soporte de programas, librerias y un lenguaje interpretado entre otros programas para ayudar a desarrollar y unir los diferentes componentes.\\

SCRUM &:& Metodología de desarrollo ágil caracterizado principalmente por: Adaptabilidad a cambios, solapamiento de fases de desarrollo y foco en resultados incrementales.\\

Sprint &:& Iteración recurrente utilizada en la metodología ágil SCRUM, corresponde a bloques temporales cortos y fijos.\\

UART &:& De sus siglas en inglés: Transmisor-Receptor Asíncrono Universal, es el dispositivo que controla los puertos y dispositivos serie.\\

UUID &:& Identificador único universal o universally unique identifier (UUID) es un número de 16 bytes (128 bits - Ejemplo: 550e8400-e29b-41d4-a716-446655440000).\\

RFCOMM &:& De sus siglas en inglés: Comunicación por radio frecuencia, es un conjunto simple de protocolos de transporte, construido sobre el protocolo L2CAP. El protocolo está basado en el estándar ETSI TS 07.10.\\

\end{tabular}

\newpage
\begin{tabular}{lcp{10.5cm}}
	
Backend &:& Capa posterior de un servicio informático, la cual no está pensada para el usuario y tiene como fin sustentar el servicio como proveer datos o lógica.\\
IP &:& Una dirección IP es un número que identifica, de manera lógica y jerárquica, a una Interfaz en red. La más común es la IPv4 aunque también existe la IPv6 y su única diferencia es la cantidad de bits empleados: 32 y 128 bits respectivamente.\\

Hosting &:& Servicio de alojamiento en la web.\\

HTTP &:& Protocolo de transferencia de hipertexto, es el protocolo de comunicación que permite las transferencias de información en la World Wide Web.\\

Instancia &:& En el contexto informático entiéndase como la "materialización" de cierta abstracción como lo puede ser un modelo o esquema.\\

Puerto &:& Número de 16 bits, usado por el protocolo host-a-host para identificar a qué protocolo de más alto nivel o programa de aplicación (proceso) debe entregar los mensajes de entrada.\\

Dominio &:& Nombre que sirve de identificador para una dirección IP. \\

DNS &:& El sistema de nombres de dominio (DNS, por sus siglas en inglés, Domain Name System) es un sistema de nomenclatura jerárquico descentralizado para dispositivos conectados a redes IP como Internet o una red privada. Este sistema asocia información variada con nombre de dominio asignado a cada uno de los participantes.\\


Proxy &:& Agente o sustituto autorizado para actuar en nombre de otra máquina o sistema.\\

TCP &:& Protocolo de control de transmisión, es uno de los protocolos fundamentales en Internet.\\

\end{tabular}

\newpage
\begin{tabular}{lcp{10.5cm}}
Socket &:& Concepto abstracto por el cual dos programas (posiblemente situados en computadoras distintas) pueden intercambiar cualquier flujo de datos, generalmente de manera fiable y ordenada..\\

WebSocket &:& Tecnología que proporciona un canal de comunicación bidireccional y full-duplex sobre un único socket TCP.\\

Baudio &:& Unidad de medida de la velocidad de transmisión de señales que se expresa en símbolos por segundo.\\

\end{tabular}
%%%%%%%%%%%%%%%%%%
%\begin{tabular}{lcp{10.5cm}}
%Relé&:& Interruptor controlado por un circuito eléctrico en el que, por medio de una bobina y un electroimán, se acciona un juego de uno o %varios contactos que permiten abrir o cerrar otros circuitos eléctricos independientes.\\
%\end{tabular}
%%%%%%%%%%%%%%%%%%