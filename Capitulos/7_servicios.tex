\chapter{Implementación de la solución de lado del cliente Android}\label{servicios}

\section{Servicios en Android}

La necesidad de utilizar servicios para el proyecto es que no necesariamente se requerirá de interacción con el usuario para ciertas tareas, como el acoplamiento automático del dispositivo y el celular, la búsqueda de servicios Bluetooth o la comunicación del servidor hacia y desde el dispositivo. Es por esta razón que un subproceso no cumple con lo requerido, ya que solo podría existir dentro del tiempo en que algún usuario haga uso de la aplicación (la mantenga abierta).

Los Servicios\cite{services} en Android se pueden comportar de tres maneras a grandes rasgos:
	
	-Servicio iniciado: Proceso en segundo plano con una tarea preestablecida y que al cabo de ejecutar se detendrá, normalmente no retorna algo y no interactúa con la actividad que lo generó.

	-Servicio de enlace: Proceso en segundo plano que se enlaza de componentes para existir, permite tareas que retornan valores y es capaz de interactuar con una o más componentes, incluso bajo comunicación entre procesos (IPC). Se destruye si la no hay componentes enlazados.

	-Servicio iniciado y de enlace: Servicio que puede ejecutarse de forma indefinida (tarea preestablecida) y que permite enlaces de componentes para interactuar y comunicarse. Es la unión de ambas características anteriores en un mismo servicio.


\newpage

\section{Implementación de servicios y su comunicación en Android en función de los requerimientos del proyecto}

Para el apartado asociado a los servicios en Android se comienzan las pruebas con el fin de conseguir las necesidades del proyecto: Autoconexión y envío de datos por internet. Ambos procesos deben ser llevados a cabo sin intervención del usuario y de forma permanente en cuanto la aplicación esté instalada y configurada a un equipo (placa principal con módulos de sensores). \newline
Para esto se trabaja con la idea de un servicio que contenga dos procesos (no necesariamente hilos, dado que hay que detallar el funcionamiento del dispositivo BLE para la comunicación de sus datos). Por esta razón se analiza el funcionamiento de las clases Services e Intent Services, en donde su diferencia radica en el objetivo de su ejecución: Services para ejecución indefinida hasta su detención manual e Intent Services para ejecución de una tarea concreta (contempla la creación propia de un hilo de ejecución definida). Luego para el tema de comunicación con el servicio existen distintas alternativas: AIDL, Binder y Messenger. AIDL usado para comunicación entre procesos de forma primitiva, Binder para comunicación solo con la aplicación contenedora del servicio (ventaja de poder acceder directamente a métodos públicos del servicio) y Messenger que es una forma de comunicación entre procesos con estructura basada en AIDL pero de mayor nivel y facilidad de uso por medio de un mensajero. Esta tarea queda en espera para trabajar con el dispositivo BLE, configurando sus parámetros e integrando su funcionamiento al módulo central de procesamiento Arduino, aunque se deja esbozado el servicio a necesitar, mezclando un servicio de la clase Services creado solo una vez para luego que las próximas ejecuciones de la aplicación se enlacen a este servicio (definiendo así un servicio indefinido contenedor y manipulador de la interacción con el bluetooth a través de IBinder).

