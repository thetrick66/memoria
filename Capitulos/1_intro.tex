\chapter{Introducción}\label{intro}

\section{Memorias multidisciplinaria}
La UTFSM ha manifestado, a través de sus planes de desarrollo y ejes estratégicos, la importancia de la formación de los estudiantes en competencias transversales, el fomento de la innovación, el emprendimiento y la vinculación con la industria. Es por esto que surge en la UTFSM el proyecto de Memorias Multidisciplinarias que propone impulsar el desarrollo de una nueva industria tecnológica a través de un programa de formación para la creación sistemática y sustentable de productos de innovación y emprendimientos ligados a tecnología.

Este proyecto de Memorias Multidisciplinarias se desarrolla a través de la proposición de un desafío el cual fue otorgado por el subgerente comercial de la empresa Sistemas Expertos, José Luis Araya. Sistemas Expertos e Ingeniería de Software (SEIS) es una empresa especialista con 10 años de experiencia en el desarrollo e implementación de soluciones tecnológicas para el área de la salud.

El desafío propuesto consiste en ¿Cómo podemos incorporar a bajo costo telemedicina a la salud pública, considerando restricciones económicas y geográficas? Para esto se conformó un equipo multidisciplinario quienes desarrollaron durante un año, un plan de negocio, pruebas de concepto y prototipado de la solución con lo cual se pretende formar un emprendimiento. 


%\newpage
%A raíz de las necesidades del desafío propuesto por la empresa Sistemas Expertos, se hace necesaria la incorporación de conocimientos en el ámbito técnico a nivel Hardware, Software, Telecomunicaciones, administración de proyectos, marketing, análisis de consumidor, prototipado y posterior encapsulamiento de la solución. Es por lo anterior que el equipo está compuesto por cuatro integrantes.

\newpage
\section{Equipo}

\subsection{Felipe Cordero}
Estudiante de último año de la carrera Ingeniería Civil Electrónica con Mención en Computadores. Ha trabajado en empresas de desarrollo de hardware embebido, tiene un gran interés por crear un emprendimiento y seguir el camino de desarrollo de hardware y software. Su interés en el desafío radica en participar de un proyecto que posee todas las fases de desarrollo de hardware con un cliente desde cero. Al estar relacionado con el área de salud y conectividad  permite aportar directamente a mejorar el sistema de salud pública en Chile. 

\subsection{Vanessa Muñoz}
Estudiante 5to año de Ingeniería Comercial, 25 años. Colaborado en actividades dentro de la universidad como PREUSM y actualmente trabajando por tercer año en la Feria de Empresas y Trabajo USM desempeñándose como Coordinadora General. La principal motivación por escoger este desafío es poder intervenir y mejorar algún área del sistema de la salud pública Chilena, dado que se ha podido presenciar la ineficiencia del servicio en distintas ocasiones. 
Decide abandonar el grupo por no cumplir los objetivos buscados para su trabajo de tesis.

\newpage
\subsection{Patricio Rodríguez}
Estudiante de último año en la carrera de Ingeniería Civil Telemática. Ha contribuido en distintos proyectos relacionados a procesamiento de imagen, análisis de redes, simulación, programación, entre otros. Se destaca por su gran motivación y tenacidad a la hora de desempeñar sus tareas, aportando al trabajo en equipo y facilitando la resolución de tareas. Su interés en el desafío recae en la necesidad de conectividad que este conlleva, además de estar ligado al área de la Salud.

\subsection{Sebastián Castillo}
Estudiante de último año en la carrera de Ingeniería en Diseño de Productos. Ha participado en actividades relacionadas al voluntariado, desarrollo de proyectos tecnológicos y conservación de la naturaleza. Se perfila como un profesional versátil, comprometido y que considera el trabajo multidisciplinario como fundamental en el desarrollo de soluciones para el mundo actual. El interés en este proyecto se debe a la posibilidad de poder impactar positivamente en la vida de gente con necesidades reales y mejorar, en cierta medida, su calidad de vida a través de la ingeniería, que muchas veces olvida el rol social que puede ejercer.

\newpage
\section{Desafío}
La empresa Sistemas Expertos ha planteado el siguiente desafío: ¿Cómo podemos incorporar a bajo costo telemedicina a la salud pública, considerando restricciones económicas y geográficas?. \newline
Apartir de lo anterior, se da cuenta de la necesidad actual de aplicar las tecnologías existentes en el ámbito de salud, permitiendo de esta forma mejorar la atención de los pacientes. Para conseguir este objetivo se espera el desarrollo de un dispositivo electrónico con capacidad de toma de datos y envío de los mismos. Así, se pueden identificar distintas aristas a considerar, como lo son: 
\begin{enumerate}
	\item 
	Tipo de enfermedades y pacientes a cubrir.
	\item 
	Tipo de sensores a emplear.
	\item 
	Tipo de tecnología de comunicación.
	\item 
	Nivel de interacción con el usuario.
\end{enumerate}

Con esto en mente, se debe tomar una decisión con respecto a las enfermedades a medir, ya que esto está ligado íntimamente a los sensores a utilizar. Entre ellos se encuentran:
\begin{enumerate}
	\item 
	Electrocardiograma.
	\item 
	Saturómetro.
	\item 
	Medidor de presión.
	\item 
	Termómetro.
\end{enumerate}

Además de lo anterior, para realizar la comunicación de estos datos de forma remota se contemplan dos alternativas:
\begin{enumerate}
	\item 
	Utilizar la infraestructura ya presente e implementada en el país, como lo son las antenas celulares conectadas directamente con el dispositivo.
	\item 
	Utilizar conexión a internet con un intermediario, como un smartphone, mediante una conexión bluetooth.
\end{enumerate}

Por último, respecto al nivel de interacción con el usuario, la empresa ha dejado expresa su necesidad de simplicidad en este desarrollo, descartando cualquier interfaz o comunicación directa entre el usuario final y el dispositivo. Si bien dependiendo de la tecnología a emplear esta sugerencia puede cambiar, en una primera instancia se mantiene esta línea de pensamiento en torno al desarrollo completo, intentando así mantener la sencillez en las distintas partes del dispositivo. Permitiendo de este modo reducir los datos a manipular, las interfaces a desarrollar y el riesgo de un mal uso por parte de los usuarios.