\chapter{Conclusiones}\label{conclusion}

Al término de este documento, se da por finalizado el trabajo asociado al programa de memorias multidisciplinarias, habiendo logrado los resultados propuestos como equipo y obteniendo en el proces o grandes experiencias, tanto personales como profesionales.

El desarrollo en equipo del proyecto puso sobre la palestra distintos obstáculos, como la coordinación de tareas, organización de reuniones, distintos roles asociados a las distintas etapas y a la pérdida de un integrante en los inicios. Pese a esto, fue una grata oportunidad el poder compartir con distintas áreas de la ingeniería, con sus puntos de vista y formas de enfrentar desafíos. En todos estos aspectos ayudó en gran medida la posibilidad de utilizar metodologías ágiles como lo es SCRUM, por adecuarse de gran manera a cambios repentinos y entregas iterativas de resultados en función de requerimientos a su vez bastante volátiles.

Una de las motivaciones presentes en este proyecto era el uso de nuevas tecnologías aplicadas en un problema de innovación, en donde poder aplicar herramientas presentes y adaptarlas a las necesidades actuales de la industria. En conjunto con esto, el hecho de poder acercarse al área de la medicina y comprender los desafíos técnicos inherentes a trabajar con pacientes, es de las cosas más enriquecedoras que se pueden rescatar.

En el ámbito técnico, se observaron las distintas complicaciones que tienen proyectos de esta índole (innovación médica), como lo es la cantidad de derivaciones necesarias para ser aprovechadas por un profesional, la capacidad de cómputo necesaria y la gran cantidad de datos concernientes a solo una medición (además de las variables que afectan a estos datos y que son tan importantes como ellos).

\newpage

El uso de tecnologías emergentes y de gran impacto fue una de las aristas visibles desde el inicio del proyecto, pero de poco sirven si no están pensadas para ser utilizadas por el común de la población. Es en este sentido, que el costo de la solución era un tema de gran relevancia a lo largo de todo el proyecto, puesto que el foco siempre se centró en ofrecer una solución a hospitales públicos en lo posible. Si bien al término del proyecto no se logra obtener un producto como tal, nunca se dejó de lado este aspecto y se fue consecuente en las distintas decisiones tomadas a lo largo del desarrollo del proyecto.

El proyecto posee un costo fijo de alrededor de 50 USD solo por parte del hardware, otros 10 USD mensuales por obtener un servidor en Chile como el ya presentado y 15 USD anuales por el dominio del sitio web, haciendo un total de 135 USD anuales como costos anuales. Con lo anterior, se visualizan a grandes rasgos los costos logrados (sin considerar horas asociadas a los profesionales a cargo o mantenimiento de los mismos) y sumado a la utilización de tecnologías como lo son las redes celulares se dan por logrados los principales objetivos del proyecto: Bajo costo y alta cobertura bajo las limitantes geográficas del país.

Entre las proyecciones que se esperaban en un inicio estaba la posibilidad de generar un emprendimiento a partir del proyecto, aspecto que se intentó con la incubadora de la Universidad pero que no pudo continuar por las razones ya expuestas en su momento. 