% ---------------------------------------------------------------------------------------
\chapter{Prototipo funcional V1}\label{proto1}
Al ver el esquemático terminado y las piezas seleccionadas se puede notar que se cumplen los requisitos propuestos en un principio. Realizando un listado de costos de componentes se obtiene un valor de $42 USD$ total. Además el costo de fabricación de una PCB de tamaño menor a $10x10[cms]$ es de aproximadamente $7 USD$. Lo que lleva a un costo total del dispositivo de $49 USD$ contando solamente componentes electrónicas.\\
Es importante destacar que esta primera versión del dispositivo es solo un comienzo en la larga etapa de desarrollo, luego de producir esta primera placa se debe realizar las pruebas correspondientes y verificación de integridad de la placa.\\
El diseño final de la PCB se puede observar en los anexos.\\
Además se aprendió a diseñar un dispositivo electrónico desde su prototipo en una placa de desarrollo, donde fue necesario leer constantemente las hojas de datos de todas las componentes a utilizar y seleccionar cuales son las mejores, que al mismo tiempo sean compatibles y coherente con el diseño que se está llevando a cabo.\\
También es importante destacar la experiencia que se obtuvo con el concurso de emprendimiento. Se aprendió mucho del mundo del emprendimiento al presentar el trabajo realizado. Además se tuvo la oportunidad de observar los trabajos propuestos por los otros emprendedores y estudiar los criterios de evaluación. Se dió mucha importancia al modelo de negocios y las alianzas estratégicas que tenía cada participante, mas que al trabajo realizado y estado actual de la idea o prototipo. Muchos de los seleccionados ya tenían potenciales clientes y alianzas estratégicas que los hacía mejores candidatos a ganar el dinero para el emprendimiento.